%% build: latexmk -pdf -pvc prez.tex

\documentclass[dvipsnames,aspectratio=169]{beamer}
\usetheme{Madrid}

%% kill footline
\setbeamertemplate{footline}[frame number]{}
\setbeamertemplate{navigation symbols}{}
%% \setbeamertemplate{footline}{}

%% bibliography
\bibliographystyle{alpha}
\setbeamerfont{bibliography item}{size=\footnotesize}
\setbeamerfont{bibliography entry author}{size=\footnotesize}
\setbeamerfont{bibliography entry title}{size=\footnotesize}
\setbeamerfont{bibliography entry location}{size=\footnotesize}
\setbeamerfont{bibliography entry note}{size=\footnotesize}
\setbeamertemplate{bibliography item}{}

%% kill ball enumeration
\setbeamertemplate{enumerate items}[circle]
\setbeamertemplate{section in toc}[circle]

%% kill block shadows
\setbeamertemplate{blocks}[rounded][shadow=false]
\setbeamertemplate{title page}[default][colsep=-4bp,rounded=true]

%% kill ball itemize
\setbeamertemplate{itemize items}[circle]

%% --------------------------------------------------------------------------------

\usepackage[utf8]{inputenc}
%% \usepackage[hidelinks]{hyperref}
\usepackage{amsmath}
\usepackage{cite}
\usepackage{amsthm}
\usepackage{amssymb}
\usepackage{amsfonts}
\usepackage{mathpartir}
\usepackage{scalerel}
\usepackage{stmaryrd}
\usepackage{bm}
\usepackage{graphicx}

%% --------------------------------------------------------------------------------

%% HoTT style composition
\makeatletter
\DeclareRobustCommand{\sqcdot}{\mathbin{\mathpalette\morphic@sqcdot\relax}}
\newcommand{\morphic@sqcdot}[2]{%
  \sbox\z@{$\m@th#1\centerdot$}%
  \ht\z@=.33333\ht\z@
  \vcenter{\box\z@}%
}
\makeatother

\newcommand{\mi}[1]{\mathit{#1}}
\newcommand{\ms}[1]{\mathsf{#1}}
\newcommand{\mbb}[1]{\mathbb{#1}}
\newcommand{\mbf}[1]{\mathbf{#1}}
\newcommand{\bs}[1]{\boldsymbol{#1}}

\newcommand{\push}{\mathsf{push}}
\newcommand{\p}{\mathsf{p}}
\newcommand{\q}{\mathsf{q}}
\newcommand{\data}{\mbf{data}}
\newcommand{\U}{\ms{U}}
\newcommand{\Set}{\ms{Set}}
\newcommand{\where}{\mbf{where}}
\newcommand{\Nat}{\ms{Nat}}
\newcommand{\zero}{\ms{zero}}
\newcommand{\suc}{\ms{suc}}
\newcommand{\Nil}{\ms{Nil}}
\newcommand{\Cons}{\ms{Cons}}
\newcommand{\El}{\ms{El}}
\newcommand{\Lift}{\ms{Lift}}
\newcommand{\lup}{\uparrow}
\newcommand{\ldown}{\downarrow}
\newcommand{\Sig}{\ms{Sig}}
\newcommand{\Code}{\ms{Code}}
\newcommand{\Tag}{\ms{Tag}}
\newcommand{\case}{\mbf{case}}
\newcommand{\of}{\mbf{of}}
\newcommand{\ttt}{\ms{tt}}
\newcommand{\blank}{{\mathord{\hspace{1pt}\text{--}\hspace{1pt}}}}
\newcommand{\ir}{{ir}}
\newcommand{\el}{{el}}
\newcommand{\ix}{{ix}}
\newcommand{\IR}{\ms{IR}}
\newcommand{\ih}{{ih}}
\newcommand{\intro}{\ms{intro}}
\newcommand{\IH}{\ms{IH}}
\newcommand{\map}{\ms{map}}
\newcommand{\elim}{\ms{elim}}
\newcommand{\inj}{\ms{inj}}
\newcommand{\tr}{\ms{tr}}
\newcommand{\fst}{\ms{fst}}
\newcommand{\snd}{\ms{snd}}
\newcommand{\IIR}{\ms{IIR}}
\newcommand{\Sigr}[1]{\lfloor #1 \rfloor}
\newcommand{\floord}[1]{\lfloor #1 \rfloor}
\newcommand{\ora}[1]{\overrightarrow{#1}}
\newcommand{\ola}[1]{\overleftarrow{#1}}
\newcommand{\ap}{\ms{ap}}
\newcommand{\Bool}{\ms{Bool}}
\newcommand{\Level}{\ms{Level}}
\newcommand{\emptycon}{\scaleobj{.75}\bullet}
\newcommand{\id}{\ms{id}}

\newcommand{\Con}{\ms{Con}}
\newcommand{\Sub}{\ms{Sub}}
\newcommand{\Ty}{\ms{Ty}}
\newcommand{\Tm}{\ms{Tm}}
\newcommand{\ext}{\triangleright}
\newcommand{\w}{\circ}
\newcommand{\lam}{\ms{lam}}
\newcommand{\app}{\ms{app}}
\newcommand{\bapp}{\$}
\newcommand{\proj}{\ms{proj}}
\newcommand{\exfalso}{\ms{exfalso}}
\newcommand{\true}{\ms{true}}
\newcommand{\false}{\ms{false}}
\newcommand{\BoolElim}{\ms{BoolElim}}
\newcommand{\fun}{\Rightarrow}
\newcommand{\SigElim}{\ms{SigElim}}
\newcommand{\Id}{\ms{Id}}
\newcommand{\refl}{\ms{refl}}
\newcommand{\J}{\ms{J}}
\newcommand{\G}{\mbb{G}}
\newcommand{\Path}{\ms{Path}}
\newcommand{\here}{\ms{here}}
\newcommand{\Ssw}{S^{*\w}}

\newcommand{\insigma}{\ms{in}\!\!-\!\!\sigma}
\newcommand{\indelta}{\ms{in}\!\!-\!\!\delta}

\newcommand{\Sb}{S^{*\circ}}
\newcommand{\Sbe}{{\floord{S^{*\circ}}\,\here}}
\newcommand{\PSbe}{{(\floord{S^{*\circ}}\,\here)}}
\newcommand{\Elintro}{\ms{El\!\!-\!\!intro}}
\newcommand{\elimbeta}{\elim\!-\!\!\beta}

\newcommand{\E}{\mbb{E}}
\newcommand{\F}{\mbb{F}}
\newcommand{\W}{\ms{W}}

%% --------------------------------------------------------------------------------

\title{Canonicity for Indexed Inductive-Recursive Types}
\author{\textbf{András Kovács}}
\institute{
  \inst%
       {University of Gothenburg \& Chalmers University of Technology}
}
\date{1st Oct 2025, Workshop in Honour of Peter Dybjer, Gothenburg}
\begin{document}


\frame{\titlepage}

\begin{frame}{Inductive-Recursive Types}

Mutual definition of an inductive type and a function acting on it.

\vspace{-1em}
\begin{alignat*}{3}
  & \mbf{mutual} \\
  & \quad\data\,\Code : \Set_0\,\where \\
  & \quad\quad \Nat' : \Code\\
  & \quad\quad \Pi' \hspace{0.8em} : (A : \Code) \to (\El\,A \to \Code) \to \Code
  & \\
  & \\
  & \quad\El : \Code \to \Set_0 \\
  & \quad\El\,\Nat'\hspace{1.5em}  = \Nat \\
  & \quad\El\,(\Pi'\,A\,B) = (a : \El\,A) \to \El\,(B\,a)
\end{alignat*}
\vspace{-0.5em}

Early and informal use by Per Martin-Löf \cite{martin1975intuitionistic,martinlof84sambin}.
\vspace{1em}

Formal syntax \& semantics developed by Peter and Anton \cite{dybjer00ir,dybjer99finite,DBLP:journals/apal/DybjerS03,DBLP:journals/jlp/DybjerS06}.
\vspace{1em}


\end{frame}

\begin{frame}{Inductive-Recursive Types}

Use-cases:
\begin{itemize}
\item Metatheory for TTs with various universe hierarchies:
  \begin{itemize}
    \item Normalization for TTs with countable universes \cite{martin1975intuitionistic,DBLP:journals/pacmpl/0001OV18,DBLP:journals/pacmpl/PujetT23,DBLP:journals/pacmpl/AbelDE23}.
    \item Consistency \cite{first-class-univ} and canonicity \cite{DBLP:journals/corr/abs-2502-20485} for notions of first-class universe levels.
  \end{itemize}
\item Others: partial functions \cite{DBLP:conf/tphol/BoveC01}, generic programming \cite{DBLP:journals/njc/BenkeDJ03,diehl2017fully}, large countable ordinals \cite{ir-ordinals,btb-ordinal}.
\end{itemize}
\vspace{1em}
\pause

\textbf{Canonicity}: can we evaluate every closed term in MLTT+IR to a value?
\vspace{1em}

IR has been supported for a long time in Agda with no canonicity issues,
but without formal proof.
\vspace{1em}
\pause

In this talk \& upcoming paper: formal proof of canonicity.
\end{frame}

\begin{frame}{How to specify an IR type?}

The type of \textbf{IR signatures} is an inductive type.
\vspace{-0.5em}
\begin{alignat*}{4}
  &\data\, \Sig\,i\,\{j\}\,(O : \Set_j) : \Set_{(i+1\,\sqcup\,j)}\,\where\\
  &\quad \iota\hspace{0.25em}  : O \to \Sig\,i\,O \\
  &\quad \sigma               : (A : \Set_i) \to (A \to \Sig\,i\,O) \to \Sig\,i\,O \\
  &\quad \delta\hspace{0.1em} : (A : \Set_i) \to ((A \to O) \to \Sig\,i\,O) \to \Sig\,i\,O
\end{alignat*}

The signature of the previous $\ms{Code}$ example:
\begin{alignat*}{4}
  & \ms{SigCode} : \Sig\,0\,\Set_0\\
  & \ms{SigCode} :\equiv \sigma\,\Bool\,\lambda\,t.\,\case\,t\,\of \\
  & \quad \true  \,\to \iota\,\Nat \\
  & \quad \false \to \delta\,\top\,\lambda\,{ElA}.\, \delta\,({ElA}\,\ttt)\,\lambda\,{ElB}.\,
      \iota\, ((x : {ElA}\,\ttt) \to {ElB}\,x)
\end{alignat*}

For each signature, we postulate type formation, term formation, elimination and computation rules,
for the described IR type.

\end{frame}


\begin{frame}{How to prove canonicity?}

Every closed term with IR type should be convertible to an IR constructor.
E.g.\ a closed term $t : \Code$ should be either $\Nat'$ or $\Pi'$.
\vspace{1em}

Type-theoretic Artin gluing \cite{coquand2018canonicity,gluing}:
\begin{itemize}
  \item A closed type is interpreted as a proof-relevant predicate over its closed terms.
        The predicate should imply canonicity.
  \item A closed term is interpreted as a predicate witness.
  \item Open types \& terms are generalized over closed substitutions.
\end{itemize}


\end{frame}

\begin{frame}{How to prove canonicity?}

\textbf{Notation:} we write $\bs{\Ty}$ for the set of closed types, $\bs{\Tm\,A}$ for sets of closed terms,
  and $\bs{\U : \Ty}$ for object-theoretic universes (omitting levels).
\vspace{1em}

\textbf{Example:} canonicity predicate for natural numbers.
\begin{alignat*}{3}
  &\data\,\Nat^\w : \Tm\,\Nat \to \Set\,\where\\
  & \quad \zero^\w : \Nat^\w\,\zero\\
  & \quad \suc^\w\,\,: (n : \Tm\,\Nat) \to \Nat^\w\,n \to \Nat^\w\,(\suc\,n)
\end{alignat*}
Generally: the canonicity predicate for an inductive type is a \alert{metatheoretic indexed inductive type}.


\end{frame}

\begin{frame}{How to prove canonicity?}

\textbf{Example:} the canonicity predicate for $\Code$ is the following \alert{metatheoretic indexed IR type}:
\begin{alignat*}{4}
  &\rlap{$\data\,\Code^\w : \Tm\,\Code \to \Set\,\where$}\\
  &\quad\Nat'^\w &&: \,\,&& \Code^\w\,\Nat'\\
  &\quad\Pi'^\w  &&: \,\,&&\{A : \Tm\,\Code\}(A^\w : \Code^\w\,A)\\
  &\quad         &&  && \{B : \Tm\,(\El\,A \to \Code)\}(B^\w : \{a : \Tm\,(\El\,a)\} \to \El^\w\,A^\w\,a \to \Code^\w\,(B\,a))\\
  &\quad         &&  && \to \Code^\w\,(\Pi'\,A\,B)\\
  &\rlap{}\\
  &\rlap{$\El^\w : \{t : \Tm\,\Code\} \to \Code^\w\,t \to (\Tm\,(\El\,t) \to \Set)$}\\
  &\rlap{$\El^\w\,\Nat'^\w \hspace{2.7em}= \Nat^\w$}  \\
  &\rlap{$\El^\w\,(\Pi'^\w\,A^\w\,B^\w)\,= \lambda\,f.\,\{a : \Tm\,A\} \to \El^\w\,A^\w\,a \to \El^\w\,B^\w\,(f\,a)$}
\end{alignat*}
If I have $t : \Tm\,\Code$, the gluing interpretation hands me an element of $\Code^\w\,t$.

\end{frame}

\begin{frame}{How to prove canonicity?}

In the object theory, let's assume general IR types specified by an internal $\Sig$ type.
\begin{itemize}
\item Previous slide: the object theory supports $\Code$, the canonicity proof involves $\Code^\w$.
\pause
\item Now: the object theory has all IR types, the canonicity proof involves all canonicity predicates.
\end{itemize}
\vspace{1em}
\pause

Interpreting IR types in the glued model:
\begin{itemize}
 \item We write $\IR\,S : \Tm\,\U$ for the IR \emph{type formation} rule, for $S : \Tm\,(\Sig\,O)$.
 \pause
 \item To interpret $\IR\,S$:
   \begin{itemize}
     \item By induction hypothesis, we get $S^\w : \Sig^\w\,S$ witnessing the canonicity of the
       signature $S$ itself.
     \item We compute a metatheoretic indexed IR signature by induction on $S^\w$.
       This yields the canonicity predicate for $\IR\,S$.
   \end{itemize}
 \pause
 \item We also need to interpret \emph{term formation}, \emph{elimination} and \emph{computation} rules.
 \item For these, we have to show that universal properties of IR types are
       preserved through the indexed IR encoding.
\end{itemize}
\vspace{1em}
\end{frame}


\begin{frame}{How to prove canonicity?}

The construction is a moderately technical ``generic programming'' exercise,
where we have to do some tricky induction over signatures.
\vspace{1em}

We use metatheoretic IR to show canonicity of object-theoretic IR. The metatheory has
to have more universes; in our case the metatheory has extra levels $\omega$ and $\omega+1$\footnote{But $\omega+1$ is only used for convenience and could be omitted.}.
\vspace{1em}

The canonicity interpretation of IR is formalized in Agda using a shallow embedding of $\Ty$ and $\Tm$.

\end{frame}

\begin{frame}{Indexed Induction-Recursion}

What about canonicity for \textbf{indexed} IR types?
\vspace{1em}

We only show canonicity for plain IR types, but also show that indexed IR types are
constructible in MLTT+IR.
\vspace{1em}

The construction supports strict computation rules for all IIR types that
are definable in Agda. For \emph{neutral} signatures we only get propositional computation.

\vspace{1em}
We use the well-known construction that converts indices to parameters and propositional identities.\footnote{Popularized as ``fording'' by Conor McBride.}

\vspace{1em}
Here, since we work in plain MLTT without function extensionality and UIP, we have to do a modest amount of HoTT reasoning.

\end{frame}

\begin{frame}{Closing notes}

\vspace{-1em}
\begin{itemize}
\item Future work:
  \begin{itemize}
    \item Normalization for IR types. For this, we first have to show that presheaf models have IR types (which is already quite technical!).
    \item Generalize the canonicity proof to a model construction (i.e.\ extend general type-theoretic gluing with IR types). This seems to
          be incompatible with first-class signatures.
  \end{itemize}
\item Paper: under review, I'll publish a preprint in a few weeks.
\end{itemize}
\vspace{3em}
\pause

\begin{center}
  \Large{\textbf{Thank you!}}
\end{center}
\end{frame}

\bibliographystyle{ACM-Reference-Format}
\bibliography{references}

\end{document}
