
%% build: latexmk -pdf -pvc paper.tex

\documentclass[acmsmall,screen,review,anonymous]{acmart}
%% \documentclass[nonacm,acmsmall]{acmart}
%% \documentclass[acmsmall]{acmart}
%% \raggedbottom

%% \BibTeX command to typeset BibTeX logo in the docs
\AtBeginDocument{%
  \providecommand\BibTeX{{%
    Bib\TeX}}}

%% Rights management information.newcomm  This information is sent to you
%% when you complete the rights form.  These commands have SAMPLE
%% values in them; it is your responsibility as an author to replace
%% the commands and values with those provided to you when you
%% complete the rights form.
\setcopyright{acmlicensed}
\copyrightyear{2018}
\acmYear{2018}
\acmDOI{XXXXXXX.XXXXXXX}

%%
%% These commands are for a JOURNAL article.
\acmJournal{JACM}
\acmVolume{37}
\acmNumber{4}
\acmArticle{111}
\acmMonth{8}

%%
%% The majority of ACM publications use numbered citations and
%% references.  The command \citestyle{authoryear} switches to the
%% "author year" style.
%%
%% If you are preparing content for an event
%% sponsored by ACM SIGGRAPH, you must use the "author year" style of
%% citations and references.
%% Uncommenting
%% the next command will enable that style.
\citestyle{acmauthoryear}

%% --------------------------------------------------------------------------------

\usepackage{xcolor}
\usepackage{mathpartir}
\usepackage{todonotes}
\presetkeys{todonotes}{inline}{}
\usepackage{scalerel}
\usepackage{bm}
\usepackage{mathtools}
\usepackage{stmaryrd}
\usepackage[title]{appendix}

\newcommand{\mit}[1]{{\mathsf{#1}}}
\newcommand{\msf}[1]{{\mathsf{#1}}}
\newcommand{\mbf}[1]{{\mathbf{#1}}}
\newcommand{\mbb}[1]{\mathbb{#1}}
\newcommand{\push}{\mathsf{push}}
\newcommand{\p}{\mathsf{p}}
\newcommand{\q}{\mathsf{q}}
\newcommand{\data}{\mbf{data}}
\newcommand{\U}{\msf{U}}
\newcommand{\Set}{\msf{Set}}
\newcommand{\where}{\mbf{where}}
\newcommand{\Nat}{\msf{Nat}}
\newcommand{\zero}{\msf{zero}}
\newcommand{\suc}{\msf{suc}}
\newcommand{\Nil}{\msf{Nil}}
\newcommand{\Cons}{\msf{Cons}}
\newcommand{\El}{\msf{El}}
\newcommand{\Lift}{\msf{Lift}}
\newcommand{\lup}{\uparrow}
\newcommand{\ldown}{\downarrow}
\newcommand{\Sig}{\msf{Sig}}
\newcommand{\Code}{\msf{Code}}
\newcommand{\Tag}{\msf{Tag}}
\newcommand{\case}{\mbf{case}}
\newcommand{\of}{\mbf{of}}
\newcommand{\ttt}{\msf{tt}}
\newcommand{\blank}{{\mathord{\hspace{1pt}\text{--}\hspace{1pt}}}}
\newcommand{\ir}{{ir}}
\newcommand{\el}{{el}}
\newcommand{\ix}{{ix}}
\newcommand{\IR}{\msf{IR}}
\newcommand{\ih}{{ih}}
\newcommand{\intro}{\msf{intro}}
\newcommand{\IH}{\msf{IH}}
\newcommand{\map}{\msf{map}}
\newcommand{\elim}{\msf{elim}}
\newcommand{\inj}{\msf{inj}}
\newcommand{\tr}{\msf{tr}}
\newcommand{\fst}{\msf{fst}}
\newcommand{\snd}{\msf{snd}}
\newcommand{\IIR}{\msf{IIR}}
\newcommand{\Sigr}[1]{\lfloor #1 \rfloor}
\newcommand{\floord}[1]{\lfloor #1 \rfloor}
\newcommand{\ora}[1]{\overrightarrow{#1}}
\newcommand{\ola}[1]{\overleftarrow{#1}}
\newcommand{\ap}{\msf{ap}}
\newcommand{\Bool}{\msf{Bool}}
\newcommand{\Level}{\msf{Level}}
\newcommand{\emptycon}{\scaleobj{.75}\bullet}
\newcommand{\id}{\msf{id}}

\newcommand{\Con}{\msf{Con}}
\newcommand{\Sub}{\msf{Sub}}
\newcommand{\Ty}{\msf{Ty}}
\newcommand{\Tm}{\msf{Tm}}
\newcommand{\ext}{\triangleright}
\newcommand{\w}{\circ}
\newcommand{\lam}{\msf{lam}}
\newcommand{\app}{\msf{app}}
\newcommand{\bapp}{\$}
\newcommand{\proj}{\msf{proj}}
\newcommand{\exfalso}{\msf{exfalso}}
\newcommand{\true}{\msf{true}}
\newcommand{\false}{\msf{false}}
\newcommand{\BoolElim}{\msf{BoolElim}}
\newcommand{\fun}{\Rightarrow}
\newcommand{\SigElim}{\msf{SigElim}}
\newcommand{\Id}{\msf{Id}}
\newcommand{\refl}{\msf{refl}}
\newcommand{\J}{\msf{J}}
\newcommand{\G}{\mbb{G}}
\newcommand{\Path}{\msf{Path}}
\newcommand{\here}{\msf{here}}
\newcommand{\Ssw}{S^{*\w}}

\newcommand{\insigma}{\msf{in}\!\!-\!\!\sigma}
\newcommand{\indelta}{\msf{in}\!\!-\!\!\delta}

\newcommand{\Sb}{S^{*\circ}}
\newcommand{\Sbe}{{\floord{S^{*\circ}}\,\here}}
\newcommand{\PSbe}{{(\floord{S^{*\circ}}\,\here)}}
\newcommand{\Elintro}{\msf{El\!\!-\!\!intro}}
\newcommand{\elimbeta}{\elim\!-\!\!\beta}

%% --------------------------------------------------------------------------------

%%
%% end of the preamble, start of the body of the document source.
%% \hypersetup{draft}
\begin{document}


\title{Canonicity for Indexed Inductive-Recursive Types}

\author{András Kovács}
\orcid{0000-0002-6375-9781}
\affiliation{%
  \institution{University of Gothenburg \& Chalmers University of Technology}
  \city{Gothenburg}
  \country{Sweden}
}
\email{andrask@chalmers.se}


\begin{abstract}
We prove canonicity for a Martin-Löf type theory that supports a countable universe hierarchy where
each universe supports indexed inductive-recursive (IIR) types. We proceed in two steps. First, we
construct IIR types from inductive-recursive (IR) types and other basic type formers, in order to
simplify the subsequent canonicity proof. The constructed IIR types support the same definitional
computation rules that are available in Agda's native IIR implementation. Second, we give a
canonicity proof for IR types, building on the well-known method of Artin gluing. The main idea is
to encode the canonicity predicate for each IR type using a metatheoretic IIR type.
\end{abstract}

%% \begin{CCSXML}
%% \end{CCSXML}
%% \ccsdesc[500]{Theory of computation~Type theory}
%% \ccsdesc[500]{Software and its engineering~Source code generation}
%% \keywords{}

\maketitle

\section{Introduction}\label{sec:introduction}

Induction-recursion (IR) was first used by Martin-Löf in an informal way \cite{TODO}, then made
formal by Dybjer and Setzer \cite{TODO}, who also developed set-theoretic and categorical semantics
\cite{TODO}. A common application of IR is to define custom universe hierarchies inside a type
theory. In the proof assistant Agda, we can use IR to define a universe that is closed under our
choice of type formers:
\begin{alignat*}{3}
  & \mbf{mutual} \\
  & \quad\data\,\Code : \Set_0\,\where \\
  & \quad\quad \Nat' : \Code\\
  & \quad\quad \Pi' \hspace{0.8em} : (A : \Code) \to (\El\,A \to \Code) \to \Code
  & \\
  & \\
  & \quad\El : \Code \to \Set_0 \\
  & \quad\El\,\Nat'\hspace{1.5em}  = \Nat \\
  & \quad\El\,(\Pi'\,A\,B) = (a : \El\,A) \to \El\,(B\,a)
\end{alignat*}
Here, $\Code$ is a type of codes of types which behaves as a custom Tarski-style universe. This
universe, unlike the ambient $\Set_0$ universe, supports an induction principle and can be used to
define type-generic functions. \emph{Indexed induction-recursion} (IIR) additionally allows indexing
$\Code$ over some type, which lets us define inductive-recursive predicates \cite{TODO}.

One application of IR has been to develop semantics for object theories that support universe
hierarchies. IR has been used in normalization proofs \cite{TODO}, in modeling first-class universe
levels \cite{TODO} and proving canonicity for them \cite{TODO}. Other applications are in
characterizing domains of partial functions \cite{TODO} and in generic programming over universes of
type descriptions or data layout descriptions \cite{TODO}.

IIR has been supported in Agda 2 since the early days of the system \cite{TODO}, and it is also
available in Idris 1 and Idris 2 \cite{TODO}. In these systems, IR has been implemented in the
``obvious'' way, supporting closed program execution in compiler backends and normalization during
type checking, but without any formal justification.

Our \textbf{main contribution} is to \textbf{show canonicity} for a Martin-Löf type theory that
supports a countable universe hierarchy, where each universe supports indexed inductive-recursive
types. Canonicity means that every closed term is definitionally equal to a canonical
term. Canonical terms are built only from constructors; for instance, a canonical natural number
term is a numeral. Hence, canonicity justifies evaluation for closed terms. The outline of our
development is as follows.

\begin{enumerate}
\item In Section \ref{TODO} we specify what it means to support IR and IIR, using Dybjer and
  Setzer's rules with minor modifications \cite{TODO}. We use first-class signatures,
  meaning that descriptions of (I)IR types are given as ordinary inductive types internally.
\item In Section \ref{TODO} we construct IIR types from IR types and other basic type formers. This
  allows us to only consider IR types in the subsequent canonicity proof, which is a significant
  simplification. In the construction of IIR types, we lose some definitional equalities when IIR
  signatures are neutral, but we still get the same computation rules that are available for Agda and
  Idris' IIR types. We formalize the construction in Agda.
\item In Section \ref{TODO}, we give a proof-relevant logical predicate interpretation of the type
  theory, from which canonicity follows. We build on the well-known method of Artin gluing
  \cite{TODO}. The main challenge here is to give a logical predicate interpretation of IR types. We
  do this by using IIR in the metatheory: from each object-theoretic signature we compute a
  metatheoretic IIR signature which encodes the canonicity predicate for the corresponding IR type.
  We formalize the predicate interpretation of IR types in Agda, using a shallow embedding of the
  syntax of the object theory. Hence, there is a gap between the Agda version and the fully formal
  construction, but we argue that it is a modest gap.
\end{enumerate}

\section{Specification for (I)IR Types}\label{sec:specification}

In this section we describe the object type theory, focusing on the specification of IR and IIR
types. We do not yet go into the formal details; instead, we shall mostly work with internal
definitions in an Agda-like syntax. In Section \ref{TODO} we will give a more rigorous specification that
is based on categories-with-families.

\subsection{Basic Type Formers}
We have a countable hierarchy of Russell-style universes, written as $\U_i$, where $i$ is an
external natural number. We have $\U_i : \U_{i + 1}$. We have $\Pi$-types as $(x : A) \to B\,x$,
which has type $\U_{\max(i,\,j)}$ when $A : \U_i$ and $B : A \to \U_j$. We use Agda-style implicit
function types for convenience, as $\{x : A\} \to B\,x$, to mark that a function argument should be
inferred from context. We have $\Sigma$-types, written as $(x : A) \times B\,x$, which also has type
$\U_{\max(i,\,j)}$. We have the unit type $\top : \U_i$ with unique inhabitant $\ttt$. We have
$\Bool : \U_0$ for Booleans. We have intensional identity types, as $t = u : \U_i$ for $t : A :
\U_i$. We define (by identity elimination) a transport operation $\tr : \{A : \U_i\}(P : A \to
\U_j)\{x\,y : A\} \to x = y \to P\,x \to P\,y$. We derive some other type formers below.
\begin{itemize}
  \item We define a universe lifting operation $\Lift : \U_i \to \U_{\max(i,\,j)}$ such that
    $\Lift\,A$ is definitionally isomorphic to $A$, by setting $\Lift\,A$ to $A \times \top_j$. We
    write the wrapping operation as $\lup\,:\,A \to \Lift\,A$ with inverse $\ldown$.
  \item We define the empty type $\bot : \U_0$ as $\true = \false$.
  \item We can define finite sum types from $\bot$, $\Sigma$ and $\Bool$. These are useful
    as ``constructor tags'' in inductive types.
\end{itemize}
We write $\blank\!\!\equiv\!\!\blank$ for definitional equality, and give definitions using $:\equiv$.

\subsection{IR Types}
The object theory additionally supports inductive-recursive types. On a high level, the specification consists
of the following.
\begin{enumerate}
\item A type of signatures. Each signature describes an IR type. Also, we internally define some
  functions on signatures which are required in the specification of other rules.
\item Rules for type formation, term formation and the recursive function, with a computation rule
  for the recursive function.
\item The induction principle with a $\beta$-rule.
\end{enumerate}

\subsubsection{IR signatures}\label{sec:ir-signatures}
Signatures are parameterized by the following data:
\begin{itemize}
\item The level $i$ is the size of the IR type that is being specified.
\item The level $j$ is the size of the recursive output type.
\item $O : \U_j$ is the output type.
\end{itemize}
IR signatures are specified by the following inductive type. We only mark $i$ and $O$ as parameters to $\Sig$,
since $j$ is inferable from $O$.
\begin{alignat*}{4}
  &\data\, \Sig_i\,O : \U_{\max(i+1,\,j)}\,\where\\
  &\quad \iota\hspace{0.25em}  : O \to \Sig_i\,O \\
  &\quad \sigma               : (A : \U_i) \to (A \to \Sig_i\,O) \to \Sig_i\,O \\
  &\quad \delta\hspace{0.1em} : (A : \U_i) \to ((A \to O) \to \Sig_i\,O) \to \Sig_i\,O
\end{alignat*}
Formally, we can view $\Sig$ in two ways: it is either a primitive inductive family \cite{TODO} or
it is defined as a $\msf{W}$-type \cite{whynotw}. The choice is not crucial, but in this paper
we treat $\Sig$ as a native inductive type, in order to avoid encoding overheads.

\begin{example}
We can reproduce the Agda example from Figure \cite{TODO}. First, we need an enumeration type to
represent the constructor labels of $\Code$. We assume this as $\Tag : \U_0$ with constructors
$\Nat'$ and $\Pi'$, and we use an informal case splitting operation for it. We also assume $\Nat :
\U_0$ for natural numbers and a right-associative $\blank\!\$\!\blank$ operator for function application.
\begin{alignat*}{4}
  & S : \Sig_0\,\U_0\\
  & S :\equiv \sigma\,\Tag\,\$\,\lambda\,t.\,\case\,t\,\of \\
  & \quad \Nat' \to \iota\,\Nat \\
  & \quad \Pi'\hspace{0.85em} \to \delta\,\top\,\$\,\lambda\,{ElA}.\, \delta\,({ElA}\,\ttt)\,\$\,\lambda\,{ElB}.\,
      \iota\, ((x : {ElA}\,\ttt) \to {ElB}\,x)
\end{alignat*}
First, we introduce a choice between two constructors by $\sigma\,\Tag$. In the $\Nat'$ branch, we
specify that the recursive function maps the constructor to $\Nat$. In the $\Pi'$ branch, we first
introduce a single inductive constructor field by $\delta\,\top$, where $\top$ represents the number
of introduced fields. The naming of the freshly bound variable ${ElA}$ is meant to suggest that it
represent the recursive function's output for the inductive field. It has type $\top \to \U_0$.
Next, we introduce $({ElA}\,\ttt)$-many inductive fields, and bind ${ElB} : {ElA}\,\ttt \to \U_0$ to
represent the corresponding recursive output. Finally, $\iota\, ((x : {ElA}\,\ttt) \to {ElB}\,x)$
specifies the output of the recursive function for a $\Pi'$ constructor.
\end{example}

Our signatures are identical to Dybjer and Setzer's \cite{TODO}, except for one difference.  We have
countable universe levels, while Dybjer and Setzer use a logical framework presentation with only
three universes, $\msf{set}$, $\msf{stype}$ and $\msf{type}$, where $\msf{set}$ contains the
inductively specified type, $\msf{stype}$ contains the non-inductive constructor arguments and
$\msf{type}$ contains the recursive output type and the type of signatures.

\subsubsection{Type and term formation}\label{sec:ir-type-and-term-formation}

In this section we also follow Dybjer and Setzer \cite{TODO}, with minor differences of notation, and
also accounting for the refinement of universe levels.

First, assuming $i$ and $O : \U_j$, a signature $S : \Sig_i\,O$ can be interpreted as a function
from $(A : \U_i) \times (A \to O)$ to $(A : \U_i) \times (A \to O)$. This can be extended to an
endofunctor on the slice category $\U_i/O$, but in the following we only need the action on
objects. We split this action to two functions, to aid readability:
\begin{alignat*}{3}
  &\rlap{$\blank_0 : \Sig_i\,O \to (\ir : \U_i) \to (\ir \to O) \to \U_i$} \\
  &S_0\,(\iota\,o)    \,&&\ir\,\el && \hspace{-0.7em}:\equiv \top\\
  &S_0\,(\sigma\,A\,S)\,&&\ir\,\el && \hspace{-0.7em}:\equiv (a : A) \times (S\,a)_0\,\ir\,\el\\
  &S_0\,(\delta\,A\,S)\,&&\ir\,\el && \hspace{-0.7em}:\equiv (f : A \to \ir) \times (S\,(\el \circ f))_0\,\ir\,\el\\
  & && &&\\
  &\rlap{$\blank_1 : (S : \Sig_i\,O) \to S_0\,\ir\,\el \to O$} \\
  &S_1\,(\iota\,o)    \,&&x       && :\equiv o\\
  &S_1\,(\sigma\,A\,S)\,&&(a,\,x) && :\equiv (S\,i)_1\,x\\
  &S_1\,(\delta\,A\,S)\,&&(f,\,x) && :\equiv (S\,(\el \circ f))_1\,x
\end{alignat*}
Although we use Agda-like pattern matching notation above, these functions are formally defined by
the elimination principle of $\Sig$. Also note the quantification of the $i$ and $j$ universe
levels. The object theory does not support universe polymorphism, so this quantification is
understood to happen in the metatheory. The introduction rules are the following.
\begin{alignat*}{3}
  &\IR                && : (S : \Sig_i\,O) \to \U_i\\
  &\El                && : \IR\,S \to O\\
  &\intro             && : S_0\,(\IR\,S)\,\El \to \IR\,S\\
  &\msf{El\!\!-\!\!intro} && : \El\,(\intro\,x) \equiv S_1\,x
\end{alignat*}
Above, we leave some rule arguments implicit, like $S$ in $\El$, $\intro$ and
$\msf{El\!\!-\!\!intro}$. The rule $\msf{El\!\!-\!\!intro}$ specifies a definitional equality.  Note
that these rules are not internal definitions but part of the specification of the object
theory. Hence, they are also assumed to be stable under object-theoretic substitution. On a high
level, the introduction rules express the existence of an $S$-algebra where we view $S$ as an
endofunctor on $\U_i/O$.

\subsubsection{Elimination}\label{sec:ir-elimination}

Here we follow the specification in \cite{TODO}. We assume another universe level $k$ that specifies
the size of the type into which we eliminate. We define two additional functions on signatures:
\begin{alignat*}{3}
  &\rlap{$\blank_\IH : (S : \Sig_i\,O)(P : \ir \to \U_k) \to S_0\,\ir\,\el \to \U_{\max(i,\,k)}$} \\
  &(\iota\,o)_\IH    \,&&P\,x       &&:\equiv \top \\
  &(\sigma\,A\,S)_\IH\,&&P\,(a,\,x) &&:\equiv (S\,a)_\IH\,P\,x \\
  &(\delta\,A\,S)_\IH\,&&P\,(f,\,x) &&:\equiv ((a : A) \to P\,(f\,a)) \times (S\,(\el \circ f))_\IH\,P\,x
\end{alignat*}
\begin{alignat*}{3}
  &\hspace{-7em}\rlap{$\blank_\map : (S : \Sig_i\,O)(P : \ir \to \U_k) \to ((x : \ir) \to P\,x) \to (x : S_0\,\ir\,\el) \to S_\IH\,P\,x$} \\
  &\hspace{-7em}(\iota\,o)_\map    \,&&P\,h\,x       &&:\equiv\,\ttt \\
  &\hspace{-7em}(\sigma\,A\,S)_\map\,&&P\,h\,(a,\,x) &&:\equiv (S\,a)_\map\,P\,h\,x \\
  &\hspace{-7em}(\delta\,A\,S)_\map\,&&P\,h\,(f,\,x) &&:\equiv (h \circ f,\,(S\,(\el \circ f))_\map\,P\,h\,x)
\end{alignat*}
$\blank_\IH$ stands for ``induction hypothesis'': it specifies having a witness of a predicate $P$ for
each inductive field in a value of $S_0\,\ir\,\el$. $S_\map$ maps over $S_0\,\ir\,\el$, applying the
section $h : (x : \ir) \to P\,x$ to each inductive field. Elimination is specified as follows.
\begin{alignat*}{3}
  &\elim           &&: (P : \IR\,S \to \U_k) \to ((x : S_0\,(\IR\,S)\,\El) \to S_\IH\,P\,x \to P\,(\intro\,x)) \to (x : \IR\,S) \to P\,x \\
  &\elim\!-\!\!\beta &&: \elim\,P\,f\,(\intro\,x) \equiv f\,x\,(S_\map\,P\,(\elim\,P\,f)\,x)
\end{alignat*}
If we have function extensionality, this specification of elimination can be shown to be equivalent
to the initiality of $(\IR\,S,\,\El)$ as an $S$-algebra \cite{TODO}.

\subsection{IIR Types}\label{sec:iir}

In IIR signatures, the sole deviation from Dybjer and Setzer is again our use of countable universe
levels \cite{TODO}. Since IIR is quite similar to IR, we present the rules without much commentary.

\subsubsection{Signatures}\label{sec:iir-signatures} We assume levels $i$, $j$, $k$, an indexing type $I : \U_k$ and a type family
for the recursive output as $O : I \to \U_j$. Signatures are as follows.
\begin{alignat*}{4}
  &\data\, \Sig_i\,I\,O : \U_{\max(i+1,\,j,\,k)}\,\where\\
  &\quad \iota\hspace{0.25em}  : (i : I) \to O\,i \to \Sig_i\,I\,O \\
  &\quad \sigma                : (A : \U_i) \to (A \to \Sig_i\,I\,O) \to \Sig_i\,I\,O \\
  &\quad \delta\hspace{0.1em}  : (A : \U_i)(\ix : A \to I) \to (((a : A) \to O\,(\ix\,a)) \to \Sig_i\,I\,O) \to \Sig_i\,I\,O
\end{alignat*}

\begin{example} We reproduce length-indexed vectors as an IIR type. We assume $A : \U_0$ for a type of elements in the vector,
and a type $\Tag : \U_0$ with inhabitants $\Nil'$ and $\Cons'$.
\begin{alignat*}{4}
  & S : \Sig_0\,\Nat\,(\lambda\,\_.\,\top)\\
  & S :\equiv \sigma\,\Tag\,\$\,\lambda\,t.\,\case\,t\,\of \\
  & \quad \Nil'\hspace{0.85em} \to \iota\,\zero\,\ttt \\
  & \quad \Cons'\to \sigma\,\Nat\,\$\,\lambda\,n.\,\sigma\,A\,\$\,\lambda\,\_.\,\delta\,\top\,(\lambda\,\_.\,n)\,\$\,\lambda\,\_.\,\iota\,(\suc\,n)\,\ttt
\end{alignat*}
We set $O$ to be constant $\top$ because vectors do not have an associated recursive function. In
the $\Nil'$ case, we simply set the constructor index to $\zero$. In the $\Cons'$ case, we introduce
a non-inductive field, binding $n$ for the length of the tail of the vector. Then, when we introduce
the inductive field using $\delta$, we use $(\lambda\,\_.\,n)$ to specify that the length of the
(single) inductive field is indeed $n$. Finally, the length of the $\Cons'$ constructor is $\suc\,n$.
\end{example}

\subsubsection{Type and term formation}\label{sec:iir-type-and-term-formation}
$\blank_0$ and $\blank_1$ are similar to before:
\begin{alignat*}{4}
  &\hspace{-6em}\rlap{$\blank_0 : \Sig_i\,I\,O \to (\ir : I \to \U_{\max(i,\,k)}) \to (\{i : I\} \to \ir\,i \to O\,i) \to I \to \U_{\max(i,\,k)}$} \\
  &\hspace{-6em}S_0\,(\iota\,i'\,o)     \,&&\ir\,\el\,i && :\equiv \Lift\,(i' = i)\\
  &\hspace{-6em}S_0\,(\sigma\,A\,S)     \,&&\ir\,\el\,i && :\equiv (a : A) \times (S\,a)_0\,\ir\,\el\,i\\
  &\hspace{-6em}S_0\,(\delta\,A\,\ix\,S)\,&&\ir\,\el\,i && :
                     \equiv (f : (a : A) \to \ir\,(\ix\,a)) \times (S\,(\el \circ f))_0\,\ir\,\el\,i\\
  &\hspace{-6em} && &&\\
  &\hspace{-6em}\rlap{$\blank_1 : (S : \Sig_i\,I\,O) \to S_0\,\ir\,\el\,i \to O\,i$} \\
  &\hspace{-6em}S_1\,(\iota\,i'\,o)\,&&\hspace{-0.8em}(\lup x) && \hspace{-0.6em}:\equiv \tr\,O\,x\,o\\
  &\hspace{-6em}S_1\,(\sigma\,A\,S)\,&&\hspace{-0.8em}(a,\,x)  && \hspace{-0.6em}:\equiv (S\,i)_1\,x\\
  &\hspace{-6em}S_1\,(\delta\,A\,S)\,&&\hspace{-0.8em}(f,\,x)  && \hspace{-0.6em}:\equiv (S\,(\el \circ f))_1\,x
\end{alignat*}
Note the transport in $\tr\,O\,x\,o$: this is necessary, since $o$ has type $O\,i'$ while the
required type is $O\,i$. The type and term formation rules are the following.
\begin{alignat*}{4}
  &\IIR               && : (S : \Sig_i\,I\,O) \to I \to \U_{\max(i,\,k)}\\
  &\El                && : \IIR\,S\,i \to O\,i\\
  &\intro             && : S_0\,(\IIR\,S)\,\El\,i \to \IIR\,S\,i\\
  &\msf{El\!\!-\!\!intro} && : \El\,(\intro\,x) \equiv S_1\,x
\end{alignat*}
\subsubsection{Elimination}\label{sec:iir-elimination} $\blank_\IH$, $\blank_\map$ and elimination are
as follows. We assume a level $l$ for the target type of elimination.
\begin{alignat*}{3}
  &\rlap{$\blank_\IH : (S : \Sig_i\,I\,O)(P : \{i : I\}\to \ir\,i \to \U_l) \to S_0\,\ir\,\el\,i \to \U_{\max(i,\,l)}$} \\
  &(\iota\,i\,o)_\IH \,&&P\,x       &&:\equiv \top \\
  &(\sigma\,A\,S)_\IH\,&&P\,(a,\,x) &&:\equiv (S\,a)_\IH\,P\,x \\
  &(\delta\,A\,\ix\,S)_\IH\,&&P\,(f,\,x) &&:\equiv ((a : A) \to P\,(f\,a)) \times (S\,(\el \circ f))_\IH\,P\,x
\end{alignat*}
\begin{alignat*}{3}
  &\hspace{-3em}\rlap{$\blank_\map : (S : \Sig_i\,I\,O)(P : \{i : I\}\to \ir\,i \to \U_l)$}\\
  &\hspace{-3em}\rlap{$\hspace{3em}\to (\{i : I\}(x : \ir\,i) \to P\,x) \to (x : S_0\,\ir\,\el\,i) \to S_\IH\,P\,x$} \\
  &\hspace{-3em}(\iota\,o)_\map    \,&&P\,h\,x       &&:\equiv\,\ttt \\
  &\hspace{-3em}(\sigma\,A\,S)_\map\,&&P\,h\,(a,\,x) &&:\equiv (S\,a)_\map\,P\,h\,x \\
  &\hspace{-3em}(\delta\,A\,S)_\map\,&&P\,h\,(f,\,x) &&:\equiv (h \circ f,\,(S\,(\el \circ f))_\map\,P\,h\,x)
\end{alignat*}
\begin{alignat*}{3}
  &\elim           &&: (P : \{i : I\}\to \IIR\,S\,i \to \U_l) \to (\{i : I\}(x : S_0\,(\IIR\,S)\,\El\,i) \to S_\IH\,P\,x \to P\,(\intro\,x))\\
  &                && \hspace{0.7em}\to (x : \IIR\,S\,i) \to P\,x \\
  &\elim\!-\!\!\beta &&: \elim\,P\,f\,(\intro\,x) \equiv f\,x\,(S_\map\,P\,(\elim\,P\,f)\,x)
\end{alignat*}

\section{Construction of IIR Types}

We proceed to construct IIR types from IR types and other basic type formers. We assume $i$, $j$,
$k$, $I : \U_k$ and $O : I \to \U_j$, and also assume definitions for IIR signatures and the four
operations ($\blank_0$, $\blank_1$, $\blank_\IH$, $\blank_\map$). The task is to define $\IR$,
$\El$, $\elim$ and $\elim\!-\!\!\beta$. We use some abbreviations in the following:
\begin{itemize}
\item $\Sig_\IIR$ abbreviates the IIR signature type $\Sig_i\,I\,O$.
\item $\Sig_\IR$ abbreviates the IR signature type $\Sig_{\max(i,\,k)}\,((i : I) \times O\,i)$.
\end{itemize}
In a nutshell, the main idea in this section is to represent IIR signatures as IR signatures
together with a well-indexing predicate on algebras. First, we define an encoding function for
signatures:
\begin{alignat*}{4}
  & \rlap{$\Sigr{\blank} : \Sig_\IIR \to \Sig_\IR$}\\
  & \Sigr{\iota\,i\,o}       &&:\equiv \hspace{0.5em} &&\iota\,(i,\,o)\\
  & \Sigr{\sigma\,A\,S}      &&:\equiv \hspace{0.5em} &&\sigma\,(\Lift\,A)\,(\lambda\,a.\,\Sigr{S\,\ldown\!a})\\
  & \Sigr{\delta\,A\,\ix\,S} &&:\equiv \hspace{0.5em} &&\delta\,(\Lift\,A)\,\$\,\lambda\,f.\\
  &  &&                                &&\sigma\,((a : A) \to \fst\,(f\,(\lup\!a)) = \ix\,a)\,\$\,\lambda\,p.\\
  &  &&                                &&\Sigr{S\,(\lambda\,a.\,\tr\,O\,(p\,a)\,(\snd\,(f\,(\lup\! a))))}
\end{alignat*}
There are two points of interest. First, the encoded IR signature has the recursive output type $(i
: I) \times O\,i$, which lets us interpret $\iota\,i\,o$ as $\iota\,(i,\,o)$. Second, in the
interpretation of $\delta$, we already need to enforce well-indexing for inductive fields, or else
we cannot recursively proceed with the translation. We solve this by adding an \emph{extra field} in
the output signature, which contains a well-indexing witness of type $((a : A) \to \fst\,(f\,(\lup\!a)) = \ix\,a)$.
This lets us continue the translation for $S$, by fixing up the return type of $f$ by a transport.

\emph{Note on prior work.} Hancock et al.\ described the same translation from small IIR
signatures to small IR signatures \cite{TODO}. However, they did not present anything more about the
reduction of IIR types to IR types.

\subsection{Type and Term Formers} We can already define the $\IIR$ and $\El$ rules for IIR types. Since the encoding of signatures
already ensures the well-indexing of inductive fields in constructors, it only remains to ensure
that the ``top-level'' index matches the externally supplied index.
\begin{alignat*}{3}
  &\IIR : \Sig_\IIR \to I \to \U_{\max(i,\,k)}                         && \El : \IIR\,S\,i \to O\,i \\
  &\IIR\,S\,i :\equiv (x : \IR\,\Sigr{S}) \times \fst\,(\El\,x) = i \hspace{3em}&& \El\,(x,\,p) :\equiv \tr\,O\,p\,(\snd\,(\El\,x))
\end{alignat*}
The following shorthand describes the data that we get when we peel off an $\intro$ from an $\IIR\,S\,i$ value:
\begin{alignat*}{4}
  &\blank_{\floord{0}} : (S : \Sig_\IR) \to I \to \U_{\max(i,\,k)}\\
  &S_{\floord{0}}\,i :\equiv (x : \floord{S}_0\,(\IR\,S)\,\El) \times \fst\,(S_1\,x) = i
\end{alignat*}
Now, we can show that $S_{\floord{0}}\,i$ is equivalent to $S_0\,(\IIR\,S)\,\El\,i$, by induction on
$S$. The induction is straightforward and we omit it here. We name the components of the equivalence
as follows:\footnote{In the Agda formalization, we
compute $\tau$ by induction on $S$, although it could be generically recovered from the other
four components as well \cite{TODO}.}
\begin{alignat*}{4}
  &\ora{S_0} &&: S_0\,(\IIR\,S)\,\El\,i \to S_{\floord{0}}\,i \\
  &\ola{S_0} &&: S_{\floord{0}}\,i \to S_0\,(\IIR\,S)\,\El\,i \\
  &\eta      &&: \forall x.\,\,\ola{S_0}\,(\ora{S_0}\,x) = x \\
  &\epsilon  &&: \forall x.\,\,\ora{S_0}\,(\ola{S_0}\,x) = x \\
  &\tau      &&: \forall x.\,\,\ap\,\ora{S_0}\,(\eta\,x) = \epsilon\,(\ola{S_0}\,x)
\end{alignat*}
This is a half adjoint equivalence \cite{TODO}. The half adjoint coherence witness $\tau$ will be
necessary shortly for rearranging some transports. Next, we show that the two $\blank_1$ operations are the same, modulo
the previous equivalence, again by induction on IIR signatures.
\[\blank_{\floord{1}} : (S : \Sig_\IIR)(x : S_0\,(\IIR\,S)\,\El\,i) \to \tr\,O\,(\snd\,(\ora{S_0}\,x))\,(\snd\,(\floord{S}_1\,(\fst\,(\ora{S_0}\,x)))) = S_1\,x\]
This lets us define the other introduction rules as well.
\begin{alignat*}{3}
  &\intro    : S_0\,(\IIR\,S)\,\El\,i \to \IIR\,S\,i && \msf{El\!\!-\!\!intro} && : \El\,(\intro\,x) \equiv S_1\,x \\
  &\intro\,x :\equiv (\intro_\IR\,(\fst\,(\ora{S_0}\,x)),\,\snd\,(\ora{S_0}\,x)) \quad\quad\quad&& \msf{El\!\!-\!\!intro} && :\equiv S_{\floord{1}}\,x
\end{alignat*}

\subsection{Elimination}

We assume a level $l$ for the elimination target. Recall the type of $\elim$:
\begin{alignat*}{3}
  &\elim :\,           &&(P : \{i : I\}\to \IIR\,S\,i \to \U_l)\\
  &                \quad\to && (f : \{i : I\}(x : S_0\,(\IIR\,S)\,\El\,i) \to S_\IH\,P\,x \to P\,(\intro\,x))\\
  &                \quad\to && (x : \IIR\,S\,i) \to P\,x
\end{alignat*}
Also recall that $x : \IIR\,S\,i$ is given as a pair of some $x : \IR\,\Sigr{S}$ and $p : \fst\,(\El\,x) = i$.
The idea here is to use IR elimination on $x : \IR\,\floord{S}$, while adjusting both $P$ and $f$ to operate
on the appropriate data. We will use the following induction motive. Note that we generalize the induction goal
over the $p$ witness.
\begin{alignat*}{3}
  &\floord{P} : \IR\,\floord{S} \to \U_{\max(k,\,l)} \\
  &\floord{P}\,x :\equiv \{i : I\}(p : \fst\,(\El\,x) = i) \to P\,(x,\,p)
\end{alignat*}
Now, we have
\begin{alignat*}{3}
  & \elim_\IR\,\floord{P} : ((x : \floord{S}_0\,(\IR\,\floord{S})\,\El) \to \floord{S}_\IH\,\floord{P}\,x \to \floord{P}\,(\intro\,x))
       \to (x : \IR\,\floord{S}) \to \floord{P}\,x.
\end{alignat*}
We adjust $f$ to obtain the next argument to $\elim_\IR\,\floord{P}$. $f$ takes $S_\IH\,P\,x$ as input,
so we need a ``backwards'' conversion:
\begin{alignat*}{3}
  & \ola{S_\IH} : \{x : S_{\floord{0}}\,i\} \to \floord{S}_\IH\,\floord{P}\,(\fst\,x) \to S_\IH\,P\,(\ola{S_0}\,x)
\end{alignat*}
This is again defined by easy induction on $S$. The induction method $\floord{f}$ is as follows.
\begin{alignat*}{3}
  &\floord{f} : (x : \floord{S}_0\,(\IR\,\floord{S})\,\El) \to \floord{S}_\IH\,\floord{P}\,x \to \floord{P}\,(\intro\,x)\\
  &\floord{f}\,x\,\ih\,p :\equiv \tr\,\bigl(\lambda\,(x,\,p).\,P\,(\intro\,x,\,p)\bigr)\,\bigl(\epsilon\,(x,\,p)\bigr)\,
                                      \bigl(f\,(\ola{S_0}\,(x,\,p))\,(\ola{S_\IH}\,\ih)\bigr)
\end{alignat*}
Thus, the definition of elimination is:
\[ \elim\,P\,f\,(x,\,p) :\equiv \elim_\IR\,\floord{P}\,\floord{f}\,x\,p \]
Only the $\beta$-rule remains to be constructed:
\[ \elim\!-\!\!\beta : \elim\,P\,f\,(\intro\,x) \equiv f\,x\,(S_\map\,P\,(\elim\,P\,f)\,x) \]
Computing definitions on the \textbf{left hand side}, we get:
\begin{alignat*}{3}
  & \tr\, &&(\lambda\,(x,\,p).\,P\,(\intro\,x,\,p))\\
  &       &&(\epsilon\,(\ora{S_0}\,x))\\
  &       &&(f\,(\ola{S_0}\,(\ora{S_0}\,x))\,(\ola{S_\IH}\,(\floord{S}_\map\,\floord{P}\,(\lambda\,x\,p.\,\elim\,P\,f\,(x,\,p))\,(\fst\,(\ora{S_0}\,x)))))
\end{alignat*}
Next, we prove by induction on $S$ that $\blank_\map$ commutes with $\ora{S_0}$:
\[ S_{\floord{\map}} : \forall\,f\,x.\,\,S_\map\,P\,(\lambda\,(x,\,p).\,f\,x\,p)\,x = \tr\,\bigl(S_\IH\,P\bigr)\,\bigl(\eta\,x\bigr)\,\bigl(\ola{S_\IH}\,(\floord{S}_\map\,\floord{P}\,f\,(\fst\,(\ora{S_0}\,x)))\bigr)\]
Using this equation to rewrite the \textbf{right hand side}, we get:
\begin{alignat*}{3}
  f\,x\,\Bigl(\tr\,\bigl(S_\IH\,P\bigr)\,\bigl(\eta\,x\bigr)\,\bigl(\ola{S_\IH}\,(\floord{S}_\map\,\floord{P}\,(\lambda\,x\,p.\,\elim\,P\,f\,(x,\,p))\,(\fst\,(\ora{S_0}\,x)))\bigr)\Bigr)
\end{alignat*}
This is now promising; on the left hand side we transport the result of $f$, while on the right hand
side we transport the argument of $f$. Now, the identification on the left is
$\epsilon\,(\ora{S_0}\,x)$, while we have $\eta\,x$ on the right. However, we have $\tau\,x :
\ap\,(\ora{S_0})\,(\eta\,x) = \epsilon\,(\ora{S_0}\,x)$, which can be used in conjunction with
standard transport lemmas to match up the two sides. This concludes the construction of IIR types.

\subsection{Strictness}
We briefly analyze the strictness of computation for constructed IIR types. Clearly, since the
construction is defined by induction on IIR signatures, we only have propositional
$\msf{El\!\!-\!\!intro}$ and $\elim\!-\!\!\beta$ in the general case, where an IIR signature can be
neutral.

However, we still support the same definitional IIR computation rules as Agda and Idris. That is
because Agda and Idris only have second-class IIR signatures. There, signatures consist of
constructors with fixed configurations of fields, where constructors are disambiguated by canonical
name tags. $\El$ applied to a constructor computes definitionally, and so does the elimination
principle when applied to a constructor. Using our IIR types, we encode Agda IIR types as follows:
\begin{itemize}
\item We have $\sigma\,\Tag\,S$ on the top to represent constructor tags.
\item In $S$, we immediately pattern match on the tag.
\item All other $\Sig$ subterms are canonical in the rest of the signature.
\end{itemize}
Thus, if we apply $\El$ or $\elim$ to a value with a canonical tag, we compute past the branching on
the tag and then compute all the way on the rest of the signature. In the Agda supplement, we provide
length-indexed vectors and the $\Code$ universe as examples for constructed IIR types with strict
computation rules.

\subsection{Mechanization}

We formalized Section \ref{TODO} in Agda. For the assumption of IR, we verbatim reproduced the
specification in Section \cite{TODO}, turning $\IR$ into an inductive type and $\El$ and $\elim$
into recursive functions. The functions are not recognized as terminating by Agda, so we disable
terminating checking for them. Alternatively, we could use rewrite rules \cite{TODO}; the two
versions are the same except that rewrite rules have a noticeable performance cost in evaluation.

One small difference is that our object theory does not have internal universe levels, so we
understand level quantification to happen in a metatheory, while in Agda we use native universe
polymorphism.

\section{Canonicity}\label{sec:canonicity}

In this section we prove canonicity for the object theory extended with IR types. First, we specify
the metatheory and the object theory in more detail.

\subsection{Metatheory}\label{sec:metatheory}

\subsubsection{Specification} The metatheory supports the following:
\begin{itemize}
  \item A countable universe hierarchy and basic type formers as described in Section \ref{TODO}.
    We write universes as $\Set_i$ instead of $\U_i$, to avoid confusion with object-theoretic
    universes.
  \item Equality reflection. Hence, in the following we will only use $\blank\!=\!\blank$ to denote
    metatheoretic equality.
  \item Universe levels $\omega$ and $\omega+1$, where $\Set_\omega : \Set_{\omega + 1}$ and $\Set_{\omega + 1}$
        is a ``proper type'' that is not contained in any universe. $\Set_\omega$ and $\Set_{\omega + 1}$ are
        also closed under basic type formers.
  \item IR types (thus IIR types as well) in $\Set_i$ when $i$ is finite.
  \item An internal type of finite universe levels. This is similar to Agda's internal type of
    finite levels, called $\Level$ \cite{TODO}. The reason for this feature is the following. The
    object theory has countable levels represented as natural numbers, and we have to interpret
    those numbers as metatheoretic levels in the canonicity model, to correctly specify sizes of
    reducibility predicates.
  \item The syntax of the object theory as a quotient inductive-inductive type \cite{TODO}, to be
    described in the following section.
\end{itemize}
\emph{Notation:} $\Lift$ is derivable the same way as we have seen, but we will make all
lifting implicit in the metatheory. In the object theory, explicit lifting is advisable, because
we talk about strict computation and canonicty, so we want to be precise about definitional
content. In the metatheory, we have equality reflection, so we can be more loose.

\subsubsection{Consistency of the metatheory}

\todo{TODO}

\subsection{The Object Theory}\label{sec:object-theory}

Informally, the object theory is a Martin-Löf type theory that supports basic type formers as
described in Section \ref{TODO} and IR types as described in Section \ref{TODO}. More formally, the
object theory is given as a quotient inductive-inductive type \cite{TODO}. The sets, operations and
equations that we give in the following together constitute the inductive signature.

\subsubsection{Core substitution calculus} The basic judgmental structure is given
as a category with families (CwF) \cite{TODO} where types are additionally annotated with levels.
Concretely, we have
\begin{itemize}
\item A category of contexts and substitutions. We have $\Con : \Set_0$ for contexts and $\Sub : \Con \to \Con \to \Set_0$
  for substitutions. The empty context $\emptycon$ is the terminal object with the unique substitution $\epsilon : \Sub\,\Gamma\,\emptycon$.
  We write $\id$ for identity substitutions and $\blank\!\circ\!\blank$ for substitution composition.
\item Level-indexed types, as $\Ty : \Con \to \Nat \to \Set_0$, together with the functorial substitution operation
      $\blank[\blank] : \Ty\,\Delta\,i \to \Sub\,\Gamma\,\Delta \to \Ty\,\Gamma\,i$.
\item Terms as $\Tm : (\Gamma : \Con) \to \Ty\,\Gamma\,i \to \Set_0$, with functorial substitution operation
  $\blank[\blank] : \Tm\,\Delta\,A \to (\sigma : \Sub\,\Gamma\,\Delta) \to \Tm\,\Gamma\,A[\sigma]$.
  \emph{Notation:} both type and term substitution binds stronger than function application, so
  for example $\Tm\,\Gamma\,A[\sigma]$ means $\Tm\,\Gamma\,(A[\sigma])$.
\item Context comprehension, consisting of a context extension operation $\blank\!\ext\!\blank : (\Gamma : \Con) \to \Ty\,\Gamma\,i \to \Con$,
  weakening morphism $\p : \Sub\,(\Gamma\ext A)\,\Gamma$, zero variable $\q : \Tm\,(\Gamma\ext A)\,A[\p]$ and substitution extension $\blank,\!\blank : (\sigma : \Sub\,\Gamma\,\Delta) \to \Tm\,\Gamma\,A[\sigma] \to \Sub\,\Gamma\,(\Delta\ext A)$, such that the following equations hold:
  \begin{alignat*}{3}
    &\p \circ (\sigma,\,t)     &&= \sigma \\
    &\q[\sigma,\,t]            &&= t \\
    &(\p,\,\q)                 &&= \id \\
    &(\sigma,\,t) \circ \delta &&= (\sigma \circ \delta,\,t[\delta])
  \end{alignat*}
  Note that a De Bruijn index $N$ is represented as $\q[\p^N]$, where $\p^N$ is $N$-fold composition
  of weakening.
\end{itemize}

\subsubsection{Universes} We have Russell-style universes, where sets of terms of universes are identified with sets of types.
Concretely, we have $\U : (i : \Nat) \to \Ty\,\Gamma\,(i + 1)$ such that $\U_i[\sigma] = \U_i$ and
$\Tm\,\Gamma\,\U_i = \Ty\,\Gamma\,i$.  This lets us implicitly convert between types and terms with
universe types. Additionally, we specify that this casting operation commutes with substitution, so
substituting $t : \Tm\,\Gamma\,\U_i$ as a term and then casting to a type is the same as first
casting and then substituting as a type. Since we omit casts and overload $\blank[\blank]$, this
rule looks trivial in our notation, but it still has to be assumed.

\subsubsection{Functions} We have $\Pi : (A : \Ty\,\Gamma\,i) \to \Ty\,(\Gamma \ext A)\,j \to \Ty\,\Gamma\,\max(i,\,j)$
such that $(\Pi\,A\,B)[\sigma] = \Pi\,A[\sigma]\,B[\sigma\circ\p,\,\q]$.  Terms are specified by
$\app : \Tm\,\Gamma\,(\Pi\,A\,B) \to \Tm\,(\Gamma\ext A)\,B$ and its definitional inverse $\lam :
\Tm\,(\Gamma\ext A)\,B \to \Tm\,\Gamma\,(\Pi\,A\,B)$. This isomorphism is natural in $\Gamma$, so we
have a substitution rule $(\lam\,t)[\sigma\circ\p,\,\q] = \lam\,t[\sigma]$.
%% We derive the traditional binary application operation as follows: $t\,\bapp\,u :=
%% (\app\,t)[\id,\,u]$. We use this as a left-associative operator.  We define non-dependent
%% functions: $A \fun B := \Pi\,A\,B[\p]$ where $A : \Ty\,\Gamma\,i$ and $B : \Ty\,\Gamma\,j$.
\\

%% \noindent\emph{Stability under substitution.} In the following we implicitly assume stability under
%% substitution for every type and term former.

\emph{Notation \& conventions.} So far we have used standard definitions, but now we develop some
notations and conventions that are more tailored to our use case. CwF combinators and De Bruijn
indices get very hard to read when get to more complicated rules like identity elimination or rules
in the specification of IR types.

\begin{itemize}
\item Assuming $t : \Tm\,\Gamma\,(\Pi\,A\,B)$ and $u : \Tm\,\Gamma\,A$, traditional binary function
  application can be derived as $(\app\,t)[\id,\,u] : \Tm\,\Gamma\,B[\id,\,u]$. We overload the
  metatheoretic whitespace operator for this kind of object-level application.
\item We may give a name to a binder (a binder can be a context extension or a $\Pi$, $\Sigma$ or
  $\lam$ binder), and in the scope of the binder all occurrence of the name is desugared to a De
  Bruijn index.  We write $\Pi$-types using the same notation as in the metatheory. For example, $(A
  : \U_i) \to A \to A$ is desugared to $\Pi\,\U_i\,(\Pi\,\q\,\q[\p])$. We also reuse the notation
  and behavior of implicit functions. We write object-level lambda abstraction as $\lam\,x.t$.
\item
  In the following we specify all type and term formers as \emph{term constants with an iterated
  $\Pi$-type}.  For example, we will specify $\Id : \Tm\,\Gamma\,((A : \U_i) \to A \to A \to \U_i)$,
  instead of abstracting over $A : \Ty\,\Gamma\,i$ and $t,\,u : \Tm\,\Gamma\,A$. In the general, the
  two flavors are inter-derivable, but sticking to object-level functions lets us consistently use
  the sugar for named binders. Also, specifying stability under substitution becomes very simple: a
  substituted term constant is computed to the same constant (but living in a possibly different
  implicit context). For example, if $\sigma : \Sub\,\Gamma\,\Delta$, then $\Id[\sigma] = \Id$ specifies
  stability under substitution. Hence, we shall omit substitution rules in the following.
\end{itemize}

\subsubsection{Sigma types} We have
\begin{alignat*}{3}
  &\Sigma        &&: \Tm\,\Gamma\,((A : \U_i) \to (A \to \U_j) \to \U_{\max(i,\,j)}) \\
  &\blank,\blank &&: \Tm\,\Gamma\,(\{A : \U_i\}\{B : A \to \U_j\}(t : A) \to B\,t \to \Sigma\,A\,B)\\
  &\fst          &&: \Tm\,\Gamma\,(\{A : \U_i\}\{B : A \to \U_j\} \to \Sigma\,A\,B \to A)\\
  &\snd          &&: \Tm\,\Gamma\,(\{A : \U_i\}\{B : A \to \U_j\}(t : \Sigma\,A\,B) \to B\,(\fst\,t))
\end{alignat*}
such that $\fst\,(t,\,u) = t$, $\snd\,(t,\,u) = u$ and $(\fst\,t,\,\snd\,u) = t$.

\subsubsection{Unit} We have $\top_i : \Tm\,\Gamma\,\U_i$ with the unique inhabitant $\ttt$.

\subsubsection{Booleans} Type formation is $\Bool : \Tm\,\Gamma\,\U_0$ with constructors $\true$ and $\false$. Elimination is as follows.
\begin{alignat*}{3}
  &\BoolElim : \Tm\,\Gamma\,((P : \Bool \to \U_i) \to P\,\true \to P\,\false \to (b : \Bool) \to P\,b)\\
  & \BoolElim\,P\,t\,f\,\true\hspace{0.2em} = t\\
  & \BoolElim\,P\,t\,f\,\false = f
\end{alignat*}

\subsubsection{Identity type}
\begin{alignat*}{3}
  &\Id   &&: \Tm\,\Gamma\,((A : \U_i) \to A \to A \to \U_i)\\
  &\refl &&: \Tm\,\Gamma\,(\{A : \U_i\}(t : A) \to \Id\,A\,t\,t)
\end{alignat*}
\begin{alignat*}{3}
  &\J : \Tm\,\Gamma\,(\{A : \U_i\}\{x : A\}(P : (y : A) \to \Id\,A\,x\,y \to \U_k)\\
  & \hspace{3.5em}\to P\,(\refl\,x) \to \{y : A\}(p : \Id\,A\,x\,y) \to P\,y\,p)&&
  &\\
  &\J\,P\,r\,(\refl\,x) = r
\end{alignat*}

\subsubsection{IR types} First, we specify the type of signatures as an inductive type. We assume levels $i$
and $j$.
\begin{alignat*}{3}
  &\Sig_i  &&: \Tm\,\Gamma\,((O : \U_j) \to \U_{\max(i+1,\,j)})\\
  &\iota   &&: \Tm\,\Gamma\,(\{O : \U_j\} \to O \to \Sig_i\,O)\\
  &\sigma  &&: \Tm\,\Gamma\,(\{O : \U_j\}(A : \U_i) \to (A \to \Sig_i\,O) \to \Sig_i\,O)\\
  &\delta  &&: \Tm\,\Gamma\,(\{O : \U_j\}(A : \U_i) \to ((A \to O) \to \Sig_i\,O) \to \Sig_i\,O)
\end{alignat*}
\begin{alignat*}{3}
  &\SigElim : \Tm\,\Gamma\,(\{O : \U_j\}(P : \Sig_i\,O \to \U_k)\\
  &           \hspace{2.2em}\to ((o : O) \to P\,(\iota\,o))\\
  &           \hspace{2.2em}\to ((A : \U_i)(S : A \to \Sig_i\,O) \to ((a : A) \to P\,(S\,a)) \to P\,(\sigma\,A\,S))\\
  &           \hspace{2.2em}\to ((A : \U_i)(S : (A \to O) \to \Sig_i\,O) \to ((f : A \to O) \to P\,(S\,f)) \to P\,(\delta\,A\,S))\\
  &           \hspace{2.2em}\to (S : \Sig_i\,O) \to P\,S)\\
  &\\
  &\SigElim\,P\,i\,s\,d\,(\iota\,o) \hspace{1.1em} = i\\
  &\SigElim\,P\,i\,s\,d\,(\sigma\,A\,S) = s\,A\,S\,(\lam\,a.\,\SigElim\,P[\p]\,i[\p]\,s[\p]\,d[\p]\,(S[\p]\,a))\\
  &\SigElim\,P\,i\,s\,d\,(\delta\,A\,S) \hspace{0.1em}= d\,A\,S\,(\lam\,f.\,\SigElim\,P[\p]\,i[\p]\,s[\p]\,d[\p]\,(S[\p]\,f))
\end{alignat*}
Note the $[\p]$ weakenings in the computation rules: $P$, $i$, $s$, $d$, and $S$ are all terms
quantified in some implicit context $\Gamma$, so when we mention them under an extra binder, we have
to weaken them. Hence, we cannot fully avoid explicit substitution operations by using named
binders. We already saw rest of the specification in Section \ref{TODO} so we only give a short
summary.
\begin{itemize}
\item $\blank_0$, $\blank_1$, $\blank_\map$ and $\blank_\IH$ are defined by $\SigElim$ and they satisfy the same definitional equations
  that we saw in Section \ref{TODO}.
\item $\IR$, $\El$, $\intro$, $\elim$ are all specified as term constants that are only parameterized over contexts and some universe levels.
\end{itemize}

\subsection{Canonicity of the Object Theory}\label{sec:canonicity-model}

On a high level, canonicity is proved by induction over the syntax of the object
theory. Since the syntax is a quotient inductive-inductive type, it supports an induction principle,
which we do not write out fully here, and only use one particular instance of it. Formally, the
induction principle takes a \emph{displayed model} as an argument, which corresponds to a bundle of
induction motives and methods, and proofs that quotient equations are respected. We could present
the current construction as a displayed model. However, we find it a bit more readable to instead
use an Agda-like notation, where we specify the resulting \emph{section} of the displayed model,
which consists of a collection of mutual functions, mapping out from the syntax, which have action
on constructors and respect all quotient equations.

\emph{Notation:} in the following we write $\Tm\,A$ to mean $\Tm\,\emptycon\,A$, and
$\Sub\,\Gamma$ to mean $\Sub\,\emptycon\,\Gamma$. This will reduce clutter since we will mostly work
with closed terms and substitutions.

We aim to define the following functions by induction on object syntax.

\begin{alignat*}{3}
  &\blank^\w : (\Gamma : \Con)      && \to \Sub\,\Gamma \to \Set_\omega\\
  &\blank^\w : (A : \Ty\,\Gamma\,i) && \to \{\gamma : \Sub\,\Gamma\}(\gamma^\w : \Gamma^\w\,\gamma) \to \Tm\,A[\gamma] \to \Set_i\\
  &\blank^\w : (\sigma : \Sub\,\Gamma\,\Delta) && \to \{\gamma : \Sub\,\Gamma\}(\gamma^\w : \Gamma^\w\,\gamma) \to \Delta^\w\,(\sigma \circ \gamma)\\
  &\blank^\w : (t : \Tm\,\Gamma\,A) && \to \{\gamma : \Sub\,\Gamma\}(\gamma^\w : \Gamma^\w\,\gamma) \to A^\w\,\gamma^\w\,t[\gamma]
\end{alignat*}

This has appeared in different flavors in previous literature. It is a proof-relevant logical
predicate interpretation, which corresponds to \emph{Artin gluing} \cite{TODO}, i.e.\ categorical
gluing over the global sections functor \cite{TODO}. The concrete formulation that we use is the
type-theoretic gluing by Kaposi, Huber and Sattler \cite{TODO}. This is a mild variation which uses
dependent type families instead of the fibered families of the categorical flavor. The
type-theoretic style becomes valuable when we get to the interpretation of more complicated type
formers, where it is easier to use than diagrammatic reasoning.





%% \subsubsection{The Global Sections Functor}

%% Above, note the quantification over closing substitutions. We could continue to explicitly write out
%% closing substitutions, but it is more clear and concise to instead refer to the \emph{global
%% sections functor} and develop some notation around it. It is a weak CwF morphism between the object
%% syntax and the ``standard'' $\Set$-based model of the object theory \cite{TODO}. Concretely, we define
%% it as follows.
%% \begin{alignat*}{3}
%%   &\G : \Con \to \Set_\omega                                        &&\G : (\sigma : \Sub\,\Gamma\,\Delta) \to \G\,\Gamma \to \G\,\Delta\\
%%   &\G\,\Gamma := \Sub\,\emptycon\,\Gamma                           &&\G\,\sigma\,\gamma := \sigma \circ \gamma \\
%%   & &&\\
%%   &\G : (A : \Ty\,\Gamma\,i) \to \G\,\Gamma \to \Set_i\hspace{3em} &&\G : (t : \Tm\,\Gamma\,A) \to (\gamma : \G\,\Gamma) \to \G\,A\,\gamma\\
%%   &\G\,A\,\gamma := \Tm\,\emptycon\,A[\gamma]                      && \G\,t\,\gamma := t[\gamma]
%% \end{alignat*}
%% $\G$ has the following properties:
%% \begin{itemize}
%% \item It is a functor, so we have $\G\,\id = (\lambda\,\gamma.\,\gamma)$ and $\G\,(\sigma \circ \delta) = \G\,\sigma \circ \G\,\delta$.
%% \item It strictly preserves type substitution and term substitution, so $\G\,(A[\sigma]) = \G\,A\,(\G\,\sigma)$ and likewise for terms.
%% \item It preserves empty and extended contexts up to isomorphism, i.e.\ $\G\,\emptycon \simeq \top_\omega$ and $\G\,(\Gamma \ext A) \simeq ((\gamma : \G\,\Gamma) \times \G\,A\,\gamma)$. \emph{Notation:} we shall omit component of these isomorphisms in the following.
%% \item It has lax preservation of universes and function types:
%% \begin{alignat*}{3}
%%   &\G_\U : \G\,\U_i\,\gamma \to \Set_i\hspace{3em} && \G_\Pi : \G\,(\Pi\,A\,B)\,\gamma \to (\alpha : \G\,A\,\gamma) \to \G\,B\,(\gamma,\,\alpha)\\
%%   &\G_\U\,A := \Tm\,\emptycon\,A                   && \G_\Pi\,f\,\alpha := f\,\alpha
%% \end{alignat*}
%% We do not have functions in the other direction. For $\U$, we cannot represent an arbitrary
%% metatheoretic $\Set$ in the object language. For $\Pi$, if we have $(\alpha : \G\,A\,\gamma) \to \G\,B\,(\gamma,\,\alpha)$, that is
%% a plain metatheoretic function between closed terms, so we have no way to recover an open term for a $\lam$ body.

%% \emph{Notation:} we shall also leave $\G_\U$ and $\G_\Pi$ implicit in the following. In particular, leaving $\G_\Pi$ implicit is like
%% having overloaded function application for $\G\,(\Pi\,A\,B)\,\gamma$.
%% \end{itemize}

\subsubsection{Interpretation of the CwF and the basic type formers}

This has been described in previous literature \cite{TODO}; in particular the code supplement to
\cite{TODO} has an Agda formalization of the canonicity model with the same universe setup and basic
type formers that we use. Therefore we only present parts here which are relevant to the IR type
interpretation, which are the CwF, universes, sigma types, functions and the unit type. The
interpretation of empty and extended contexts is as follows.
\begin{alignat*}{3}
  &\emptycon^\w\,&&\gamma                 &&:= \top \\
  &(\Gamma\ext A)^\w\,&&(\gamma,\,\alpha) &&:= (\gamma^\w : \Gamma^\w\,\gamma) \times A^\w\,\gamma^\w\,\alpha
\end{alignat*}
This says that the logical predicate holds for a closing substitution if it holds for each term in
the substitution. Note the pattern matching notation in $(\gamma,\,\alpha)$: this is justified,
since all values of $\Sub\,(\Gamma\ext A)$ are uniquely given as a pairing (similarly to pattern
matching notation for plain $\Sigma$-types). The other CwF operations are as follows.
\begin{alignat*}{4}
  &\id^\w\,\gamma^\w                   &&:= \gamma^\w                  &&(\sigma,\,t)^\w\,\gamma^\w          &&:= (\sigma^\w\,\gamma^\w,\,t^\w\,\gamma^\w)\\
  &(\sigma \circ \delta)^\w\,\gamma^\w &&:= \sigma^\w\,(\delta^\w\,\gamma^\w) && \p^\w\,(\gamma^\w,\,\alpha^\w)      &&:= \gamma^\w\\
  &(A[\sigma]) ^\w\,\gamma^\w\,\alpha  &&:= A^\w\,(\sigma^\w\,\gamma^\w)\,\alpha\hspace{3em} && \q^\w\,(\gamma^\w,\,\alpha^\w)      &&:= \alpha^\w\\
  &(t[\sigma]) ^\w\,\gamma^\w          &&:= t^\w\,(\sigma^\w\,\gamma^\w) && \epsilon^\w\,\gamma^\w              &&:= \ttt
\end{alignat*}
The interpretation of \textbf{universes} is the following.
\begin{alignat*}{4}
  &(\U_i)^\w\,\gamma^\w\,\alpha := \Tm\,\alpha \to \Set_i
\end{alignat*}
This definition also supports the Russell universe rules. For illustration, assuming $t :
\Tm\,\Gamma\,\U_i$, we have $t^\w : \{\gamma : \Sub\,\Gamma\}(\gamma^\w : \Gamma^\w\,\gamma) \to
\Tm\,t[\gamma] \to \Set_i$. If we first cast $t$ to a type using the syntactic Russell
universe equation, then $t^\w$ has exactly the same type.

Note that we do not get a canonicity statement about types themselves, i.e.\ we do not get that
every closed type is definitionally equal to one of the canonical type formers. This could be handled
as well, but we skip it because it is orthogonal to the focus of this paper.

We interpret \textbf{functions} as follows.
\begin{alignat*}{4}
  &(\Pi\,A\,B)^\w\,\{\gamma\}\,\gamma^\w\,f &&:= \{\alpha : \Tm\,A[\gamma]\}(\alpha^\w : A^\w\,\gamma^\w\,\alpha) \to B^\w\,(\gamma^\w,\,\alpha^\w)\,(f\,\alpha)\\
  &(\lam\,t)^\w\,\gamma^\w &&:= \lambda\,\{\alpha\}\,\alpha^\w.\,t^\w\,(\gamma^\w,\,\alpha^\w)\\
  &(\app\,t)^\w\,(\gamma^\w,\,\alpha^\w) &&:= t^\w\,\gamma^\w\,\alpha^\w
\end{alignat*}
In the $\Pi\,A\,B$ case, note that $f : \Tm\,(\Pi\,A\,B)[\gamma]$, which means that we can apply it
to $\alpha$ to get $f\,\alpha : \Tm\,B[\gamma,\,\alpha]$.

For \textbf{$\Sigma$-types}, we have
\begin{alignat*}{4}
  &\Sigma^\w\,\gamma^\w\,A^\w\,B^\w\,(t,\,u)   &&:= (t^\w : A^\w\,t) \times B^\w\,t^\w\,u\hspace{2em}  && \fst^\w\,\gamma^\w\,(t^\w,\,u^\w) &&:= t^\w \\
  &(\blank,\blank)^\w\,\gamma^\w\,t^\w\,u^\w   &&:= (t^\w,\,u^\w)                                     && \snd^\w\,\gamma^\w\,(t^\w,\,u^\w) &&:= u^\w
\end{alignat*}
For the \textbf{unit type}, we have $\top^\w\,\gamma^\w\,t := \top$ and $\ttt^\w\,\gamma^\w := \ttt$.

\subsubsection{Interpretation of IR signatures}

Signatures are given as a particular parameterized inductive type, so in principle there should be
nothing ``new'' in their logical predicate interpretation. We do detail it here because several
later constructions depend on it. Recall that $\Sig_i : \Tm\,\Gamma\,((O : \U_j) \to \U_{\max(i+1,\,j)})$, so we have
\begin{alignat*}{3}
  &(\Sig_i)^\w\,\gamma^\w : ((O : \U_j) \to \U_{\max(i+1,\,j)})^\w\,\gamma^\w\,\Sig_i\\
  &(\Sig_i)^\w\,\gamma^\w : \{O : \Tm\,\U_j\}(O^\w : (\U_j)^\w\,\gamma^\w\,O) \to (\U_{\max(i+1,\,j)})^\w\,\gamma^\w\,(\Sig_i\,O)\\
  &(\Sig_i)^\w\,\gamma^\w : \{O : \Tm\,\U_i\}(O^\w : \Tm\,O \to \Set_i) \to \Tm\,(\Sig_i\,O) \to \Set_{\max(i+1,\,j)}.
\end{alignat*}
Hence, we define an inductive type in the metatheory that is parameterized by $O : \Tm\,U_i$ and
$O^\w : \Tm\,O \to \Set_i$ and indexed over $\Tm\,(\Sig_i\,O)$. We name this inductive type
$\Sig^\w$; the naming risks some confusion, but we shall take the risk and we will shortly
explain the rationale.
\begin{alignat*}{4}
  &\rlap{$\data\,\Sig^\w\,\{O : \Tm\,U_j\}\,(O^\w : \Tm\,O \to \Set_i)  : \Tm\,(\Sig_i\,O) \to \Set_{\max(i+1,\,j)}$}\\
  &\quad \iota^\w  &&:\,&&\{o &&: \Tm\,O\}(o^\w : O^\w\,o) \to \Sig^\w\,O^\w\,(\iota\,o)\\
  &\quad \sigma^\w &&:\,&&\{A &&: \Tm\,\U_i\}(A^\w : \Tm\,A \to \Set_i)\\
  &               && &&\{S &&: \Tm\,(A \to \Sig_i\,O)\}\\
  &               && && (S^\w &&: \{a : \Tm\,A\} \to A^\w\,a \to \Sig^\w\,O^\w\,(S\,a))\\
  &               && && \to &&\,\Sig^\w\,O^\w\,(\sigma\,A\,S)\\
  &\quad \delta^\w &&:\,&&\{A &&: \Tm\,\U_i\}(A^\w : \Tm\,A \to \Set_i)\\
  &               &&   && \{S &&: \Tm\,((A \to O) \to \Sig_i\,O)\}\\
  &               &&   && (S^\w &&: \{f : \Tm\,(A \to O)\} \to (\{a : \Tm\,A\} \to A^\w\,a \to O^\w\,(f\,a)) \to \Sig^\w\,O^\w\,(S\,f))\\
  &\quad          &&   && \to &&\,\Sig^\w\,O^\w\,(\delta\,S\,f)
\end{alignat*}
A witness of $\Sig^\w\,O^\w\,t$ tells us that $t$ is a canonical constructor and it only contains
canonical data, inductively. Now, we define $(\Sig_i)^\w\,\gamma^\w\,O^\w\,t$ to be
$\Sig^\w\,O^\w\,t$, and each syntactic $\Sig$ constructor is interpreted using the corresponding
semantic constructor. For instance:
\begin{alignat*}{4}
  &\iota^\w : \{\gamma : \Sub\,\Gamma\}(\gamma^\w : \Gamma^\w\,\gamma)\{O : \Tm\,\U_j\}\{O^\w : \Tm\,O \to \Set_j\}\{o : \Tm\,O\} \to O^\w\,o \to \Sig^\w\,O^\w\,(\iota\,o)\\
  &\iota^\w\,\gamma^\w\,o^\w := \iota^\w\,o^\w
\end{alignat*}
We skip the interpretation of the other constructors and the eliminator here. Above on the left side
we use $\iota^\w$ for specifying the action of $\blank^\w$ on the syntactic $\iota$, while on the
right side we use the metatheoretic $\Sig^\w$ constructor $\iota^\w$. In general, the recipe is:
\begin{enumerate}
\item We first give semantic interpretation of object constructions while only referring to \emph{closed terms}.
\item Then, we ``contextualize'' the definitions to get interpretations of object-theoretic rules.
\end{enumerate}
In this section, the bulk of the work in interpreting IR types only needs to refer to closed terms,
and the final step of contextualization is fairly trivial. Hence, we optimize our notation for the first
phase.

\subsubsection{Interpretation of IR types}

The basic idea is that for each IR type, the corresponding canonicity predicate should be defined as
a metatheoretic IIR type. This gets rather technical, so first let us look at an informal example
for a concrete IR type.

\begin{example}
Consider the Agda IR example in \ref{TODO}. We present the logical predicate interpretation for the
IR type in an informal Agda-like syntax, focusing on readability. We assume the same IR type in the object
theory, in Agda's style, disregarding elimination for now:
\begin{alignat*}{4}
  &\Code &&: \Tm\,\U_0 && \El : \Tm\,(\Code \to \U_0) \\
  &\Nat' &&: \Tm\,\Code && \El\,\Nat'\hspace{1.5em} = \Nat \\
  &\Pi'  &&: \Tm\,((A : \Code) \to (\El\,A \to \Code) \to \Code)\hspace{1em} && \El\,(\Pi'\,A\,B) = (a : \El\,A) \to \El\,(B\,a)
\end{alignat*}
We assume $\Nat^\w : \Tm\,\Nat \to \Set_0$. The canonicity interpretation is given by the following IIR type.
\begin{alignat*}{4}
  &\Code^\w &&: \,\,&& \Tm\,\Code \to \Set_0\quad\\
  &\El^\w   &&: \,\,&& \{t : \Tm\,\Code\} \to \Code^\w\,t \to (\Tm\,(\El\,t) \to \Set_0)\\
  &\Nat'^\w &&: \,\,&& \Code^\w\,\Nat'\\
  &\Pi'^\w  &&: \,\,&&\{A : \Tm\,\Code\}(A^\w : \Code^\w\,A)\{B : \Tm\,(\El\,A \to \Code)\}\\
  &         &&  && (B^\w : \{a : \Tm\,(\El\,a)\} \to \El^\w\,A^\w\,a \to \Code^\w\,(B\,a)) \to \Code^\w\,(\Pi'\,A\,B)\\
  &\rlap{}\\
  &\rlap{$\El^\w\,\Nat'^\w\,t \hspace{2.55em}= \Nat^\w\,t$}  \\
  &\rlap{$\El^\w\,(\Pi'^\w\,A^\w\,B^\w)\,f = \{a : \Tm\,A\} \to \El^\w\,A^\w\,a \to \El^\w\,B^\w\,(f\,a)$}
\end{alignat*}
We could also extend this with elimination for $\Code$ and then use $\Code^\w$-elimination to show
that it preserves predicates. In other words, the above definition is sufficient to prove canonicity
for $\Code$ as a concrete IR type. The task in this section is to do the same construction generically
for all IR types.
\end{example}

We proceed to the semantic definitions. We assume the following parameters: $O : \Tm\,U_j$, $O^\w :
\Tm\,O \to \Set_j$, $S^* : \Tm\,(\Sig_i\,O)$ and $S^{*\w} :
\Sig^\w\,O^\w\,S^*$. \emph{Abbreviation:} we write $\Sig^\w\,S$ in the following, omitting the fixed
$O^\w$ parameter from the type.  We view $S^*$ as a ``fixed'' top-level signature, in contrast to
``varying'' signatures that we will encounter in constructions. More concretely, we will do most
constructions by induction on \emph{canonical sub-signatures} of $S^*$.

\paragraph{Canonical sub-signatures of $S^*$.} We define an inductive family indexed
over $S$ and $S^\w : \Sig^\w\,O^\w\,S$, which represents paths into $S^*$ that lead to $S$, viewing
$S$ as a subtree. Also, the subtree $S$ and all data in the path must be canonical (i.e.\ have $\blank^\w$
witnesses). The path is represented as a left-associated \emph{snoc-list} of data that can be
plugged into $\sigma$ and $\delta$ constructors. Moreover, we restrict the $\delta$ case, only
allowing $f : \Tm\,(A \to \IR\,S^*)$ functions instead of functions of type $\Tm\,(A \to O)$.
\begin{alignat*}{4}
  &\rlap{$\data\,\Path : \{S : \Tm\,(\Sig_i\,O)\} \to \Sig^\w\,S \to \Set_{\max(i+1,\,j+1)}$}\\
  &\quad \here    &&: \Path\,\Ssw\\
  &\quad \insigma &&: \Path\,(\sigma^\w\,A^\w\,S^\w) \to \{a : \Tm\,A\}(a^\w : A^\w\,a) \to \Path\,(S^\w\,a^\w) \\
  &\quad \indelta &&: \Path\,(\delta^\w\,A^\w\,S^\w) \to \{f : \Tm\,(A \to \IR\,S^*)\}(f^\w : \{a : \Tm\,A\} \to A^\w\,a \to O^\w\,(\El\,(f\,a)))\\
  &               && \hspace{0.6em} \to \Path\,(S^\w\,f^\w)
\end{alignat*}

If we have a path to $S^\w : \Sig^\w\,S$, we can push the terms contained in the path onto a
term of $S_0\,(\IR\,S^*)\,\El$:
\begin{alignat*}{4}
  & \hspace{-6em}\rlap{$\push_0 : \Path\,S^\w \to \Tm\,(S_0\,(\IR\,S^*)\,\El) \to \Tm\,((S^*)_0\,(\IR\,S^*)\,\El)$}\\
  & \hspace{-6em}\push_0\,\here\,                     &&t := t\\
  & \hspace{-6em}\push_0\,(\insigma\,p\,\{a\}\,a^\w)\,&&t := \push_0\,p\,(a,\,t)\\
  & \hspace{-6em}\push_0\,(\indelta\,p\,\{f\}\,f^\w)\,&&t := \push_0\,p\,(f,\,t)
\end{alignat*}
We also show that this operation preserves $\blank_1$, so we have $S_1\,t = (S^*)_1\,(\push_0\,p\,t)$.

\paragraph{Encoding for signatures.} Next, we define the encoding function for signatures.
Note that this computation is only possible by induction on $\Sig^\w\,S$ \--- we have no appropriate
induction principle for $\Tm\,(\Sig_i\,O)$.  As we recurse into a signature, we store the
data that we have seen in a $\Path$, and when we hit the base case $\iota^\w$, we use the $\Path$ to
build up the correct term for the constructor index.
\begin{alignat*}{5}
  &\rlap{$\floord{\blank} : (S^\w : \Sig^\w\,S) \to \Path\,S^\w \to \Sig_\IIR\,(\Tm\,(\IR\,S^*))\,(O^\w\circ \El)$}\\
  &\floord{\iota^\w\,o^\w}\,&&p               &&:=\,&&\iota\,(\intro\,(\push_0\,p\,\ttt))\,o^\w\\
  &\floord{\sigma^\w\,\{A\}\,A^\w\,S^\w}\,&&p &&:=\,&&\sigma\,(\Tm\,A)\,\$\,\lambda\,a.\,\sigma\,(A^\w\,a)\,\$\,\lambda\,a^\w.\,\floord{S^\w\,a^\w}\,(\insigma\,p\,a^\w)\\
  &\floord{\delta^\w\,\{A\}\,A^\w\,S^\w}\,&&p &&:=\,&&\sigma\,(\Tm\,(A \to \IR\,S^*))\,\$\,\lambda\,f.\,\delta\,((a : \Tm\,A) \times A^\w\,a) (\lambda\,(a,\,\_).\,f\,a)\,\$\,\lambda\,f^\w.\\
  &                       &&              &&   &&\hspace{0.8em}\floord{S^\w\,(\lambda\,\{a\}\,a^\w.\,f^\w\,(a,\,a^\w))}\,(\indelta\,p\,(\lambda\,\{a\}\,a^\w.\,f^\w\,(a,\,a^\w)))
\end{alignat*}
Remarks.
\begin{itemize}
  \item The metatheoretic IIR type is indexed over $\Tm\,(\IR\,S^*)$, and the recursive output type
    is given by $O^\w\circ \El : \Tm\,(\IR\,S^*) \to \Set_j$. Here, we implicitly cast the syntactic
    $\El : \Tm\,(\IR\,S^* \to O)$ to the funtion type $\Tm\,(\IR\,S^*) \to \Tm\,O$.
  \item In the $\iota^\w$ case, we have $o^\w : O^\w\,o$ and
    \[ \iota : (t : \Tm\,(\IR\,S^*)) \to O^\w\,(\El\,t) \to \Sig_\IIR\,(\Tm\,(\IR\,S^*))\,(O^\w\circ \El). \]
    We have $\intro\,(\push_0\,p\,\ttt) : \Tm\,(\IR\,S^*)$. If we apply $\El$ to this term, it computes
    to $(S^*)_1\,(\push_0\,p\,\ttt)$, which is the same as $(\iota\,o)_1\,\ttt$, which is the same
    as $o$, which makes $o^\w : O^\w\,o$ well-typed for the second argument.
  \item In the $\sigma^\w$ case, we use two $\sigma$-s to abstract over a term and a canonicity witness for it.
  \item In the $\delta^\w$ case, we abstract over $f : \Tm\,(A \to \IR\,S^*)$, then we use $\delta$ to specify inductive
        witnesses for all ``subtrees'' that are obtained by applying $f$ to canonical terms.
\end{itemize}

\begin{example} Let $S^*$ be the signature from Example \ref{TODO}. It depends on the $\Tag$ and $\Nat$ types,
so we assume evident $\blank^\w$ interpretations for them. Now, $S^* : \Tm\,(\Sig_0\,\U_0)$ is a closed term
that does not refer to any IR type/term former, so we can already fully compute the $\blank^\w$
operation on it, obtaining $\Ssw : \Sig^\w\,(\lambda\,A.\,\Tm\,A \to \Set_0)\,S^*$. Then, we compute the following.
\begin{alignat*}{4}
  & \floord{\Ssw}\,\here : \Sig_\IIR\,(\Tm\,(\IR\,S^*))\,(\lambda\,t.\,\Tm\,(\El\,t) \to \Set_0)\\
  & \floord{\Ssw}\,\here = \sigma\,(\Tm\,\Tag)\,\$\,\lambda\,t.\,\sigma\,(\Tag^\w\,t)\,\$\,\lambda\,t^\w.\,\case\,t^\w\,\of\\
  & \quad \Nat'^\w \to \iota\,(\intro\,(\Nat',\,\ttt))\,\Nat^\w\\
  & \quad \Pi'^\w\hspace{0.8em} \to \sigma\,(\Tm\,(\top \to \IR\,S^*))\hspace{1.75em}\$\,\lambda\,A.\,
                      \delta\,((a : \Tm\,\top) \times \top)\hspace{4.4em}(\fst \circ A)\,\$\,\lambda\,\{{ElA}\}\,{ElA}^\w.\\
  & \hspace{4.8em} \sigma\,(\Tm\,({ElA}\,\ttt \to \IR\,S^*))\,\$\,\lambda\,B.\,
                      \delta\,((a : \Tm\,({ElA}\,\ttt)) \times {ElA}^\w\,a)\,(\fst \circ B)\,\$\,\lambda\,\{{ElB}\}\,{ElB}^\w.\\
  & \hspace{4.8em} \iota\,\,(\Pi'\,A\,B)\,(\lambda\,f.\,\{a : \Tm\,({ElA}\,\ttt)\} \to {ElA}^\w\,a \to {ElB}^\w\,(f\,a))
\end{alignat*}
This is essentially the same signature that we had in Example \ref{TODO}, with some extra noise in the first argument of $\Pi'$,
which is represented as a function with $\top$ domain.
\end{example}

\paragraph{Interpretation of $\IR$ and $\El$.}
This time around, encoded signatures get us precisely what we want:
\begin{alignat*}{4}
  &\IR^\w : \Tm\,(\IR\,S^*) \to \Set_i \quad\quad && \El^\w : \{t : \Tm\,(\IR\,S^*)\} \to \IR^\w\,t \to O^\w\,(\El\,t)\\
  &\IR^\w := \IIR_\PSbe                            && \El^\w := \El_\IIR
\end{alignat*}

\paragraph{Interpretation of $\intro$.}
For this, we need to show an equivalence between two representations of $\IR^\w$'s data, somewhat
similarly to as in Section \ref{TODO}. For $\intro$, we only need one component map of the
equivalence, but later we will need all of it.

First, we define the predicate interpretations of $\blank_0$ and $\blank_1$. The general form states
that $\blank_0$ and $\blank_1$ preserve predicates, but we will only need the special case when the
$\ir$ and $\el$ arguments are $\IR\,S$ and $\El$ respectively.
\begin{alignat*}{3}
  & \blank_{0^\w} : \Sig^\w\,S \to \Tm\,(S_0\,(\IR\,S^*)\,\El) \to \Set_i\\
  & \blank_{1^\w} : (S^\w : \Sig^\w\,S)\{t : \Tm\,(S_0\,(\IR\,S^*)\,\El)\} \to (S^\w)_{0^\w}\,t \to O^\w\,(S_1\,t)
\end{alignat*}
Second, we define
\begin{alignat*}{3}
  &\blank_{\floord{0}} : (S^\w : \Sig^\w\,S) \to \Path\,S^\w \to \Tm\,(\IR\,S^*) \to \Set_i\\
  &(S^\w)_{\floord{0}}\,p\,t := (t' : \Tm\,(S_0\,(\IR\,S^*)\,\El)) \times ((\intro\,(\push_0\,p\,t') = t) \times (S^\w)_{0^\w}\,t').
\end{alignat*}
Next, we show the following equivalence by induction on $S^\w$:
\[  (S^\w : \Sig^\w\,S)(p : \Path\,S^\w)\{t : \Tm\,(\IR\,S^*)\} \to (S^\w)_{\floord{0}}\,p\,t \simeq (\floord{S^\w}\,p)_0\,t \]
We write $\ora{(S^\w)_0}\,p$ for the map with type $(S^\w)_{\floord{0}}\,p\,t \to (\floord{S^\w}\,p)_0\,t$ and
$\ola{(S^\w)_0}\,p$ for its inverse. This lets us interpret $\intro$.
\begin{alignat*}{4}
  & \intro^\w : \{t : \Tm\,((S^*)_{0^\w}\,(\IR\,S^*)\,\El)\} \to (\Ssw)_{0^\w}\,t \to \IR^\w\,(\intro\,t)\\
  & \intro^\w\,\{t\}\,t^\w := \intro_\IIR\,(\ora{(\Ssw)_0}\,\,\here\,(t,\,\refl,\,t^\w))
\end{alignat*}

\paragraph{Interpretation of $\Elintro$.}
Similarly as in Section \ref{TODO}, we need to show that $\blank_1$ commutes with signature
encoding. For this, we need to annotate $\Path$ with additional information.  Recall that the
current definition of $\Path$ is not quite the most general notion of ``path'' in signatures,
because the $\indelta$ constructor restricts the stored syntactic functions to the form $\El \circ f
: \Tm\,(A \to O)$, only storing $f : \Tm\,(A \to \IR\,S^*)$. This restriction is required for the
definition of $\push_0$, where we need to produce $\Tm\,((S^*)_0\,(\IR\,S^*)\,\El)$ as output.

Now we also need to restrict the $f^\w$ witnesses in $\indelta$ to the form $f^\w \circ \El^\w$,
where $f^\w : \{a : \Tm\,A\} \to A^\w\,a \to \IR^\w\,(f\,a)$. We define a predicate over $\Path$
that expresses this:
\[ \msf{restrict} : \Path\,S^\w \to \Set_{\max(i+1,\,j+1)} \]
This is required for the predicate interpretation of $\push_0$, which is defined by induction on $\Path$:
\[ \push_{0^\w} : (p : \Path\,S^\w) \to \msf{restrict}\,p \to (S^\w)_{0^\w}\,t \to (\Ssw)_{0^\w}\,(\push_0\,p\,t) \]
This operation preserves $\blank_{1^\w}$:
\[ (S^\w)_{1^\w}\,t^\w = (\Ssw)_{1^\w}\,(\push_{0^\w}\,p\,q\,t^\w)  \]
We use $\push_{0^\w}$ in the statement of $\blank_{\floord{1}}$, which we prove by induction on
$S^\w$:
\[ \blank_{\floord{1}} : \forall\,S^\w\,p\,q\,t^\w.\,(\floord{S^\w}\,p)_1\,(\ora{(S^\w)_0}\,p\,t^\w) = (\Ssw)_{1^\w}\,(\push_{0^\w}\,p\,q\,t^\w) \]
Finally, we define $\Elintro^\w$:
\begin{alignat*}{3}
    &\Elintro^\w : \{t : \Tm\,((S^*)_0\,(\IR\,S^*)\,\El)\}(t^\w : (\Ssw)_{0^\w}\,t) \to \El^\w\,(\intro^\w\,t^\w) = (\Ssw)_{1^\w}\,t^\w\\
    &\Elintro^\w\,t^\w := (\Ssw)_{\floord{1}}\,\here\,\ttt\,t^\w
\end{alignat*}
Above, $\ttt$ witnesses the restriction of $\here$, which is trivial (since $\here$ does not contain $\indelta$).

\paragraph{Interpretation of $\elim$.}
We assume the following parameters to elimination:
\begin{alignat*}{4}
  &k    &&:\,\,&&\Nat\\
  &P    &&:&& \Tm\,(\IR\,S^* \to \U_k)\\
  &P^\w &&:&& \{t\} \to \IR^\w\,t \to \Tm\,(P\,t) \to \Set_k
\end{alignat*}
We define the predicate interpretations for $\blank_\IH$ and $\blank_\map$ first,
specializing the $\ir$ and $\el$ arguments to $\IR\,S^*$ and $\El$ and the target level to $k$.
\begin{alignat*}{3}
  &\blank_{\IH^\w}  &&:\,\,&& \{S\}(S^\w : \Sig^\w\,S)\{t\} \to (S^\w)_{0^\w}\,t \to \Tm\,(S_\IH\,t) \to \Set_{\max(i,\,k)}\\
  &\blank_{\map^\w} &&:\,\,&& \{S\}(S^\w : \Sig^\w\,S)\{t\}(t^\w : (S^\w)_{0^\w}\,t)\{f\}(f^\w : \{t\}(t^\w : (S^\w)_{0^\w}\,t) \to P^\w\,t^\w\,(f\,t))\\
  &               &&      &&\to (S^\w)_{\IH^\w}\,S^\w\,t^\w\,(S_\map\,f\,t)
\end{alignat*}
We also assume the induction method and its canonicity witness as parameters:
\begin{alignat*}{3}
  &f    &&:\,\,&& (t : \Tm\,((S^*)_0\,(\IR\,S^*)\,\El) \to (S^*)_\IH\,t \to P\,(\intro\,t))\\
  &f^\w &&:\,\,&& \{t\}(t^\w : (\Ssw)_{0^\w}\,t)\{\ih\} \to (\Ssw)_{\IH^\w}\,t^\w\,\ih \to P^\w\,(\intro^\w\,t^\w)\,(f\,t\,\ih)
\end{alignat*}
The goal is the following:
\[ \elim^\w : \{t\}(t^\w : \IR^\w\,t) \to P^\w\,t^\w\,(\elim\,S^*\,P\,f\,t)  \]
We shall use IIR elimination on $t^\w$ to give the definition. Again like in Section \ref{TODO}, we
have to massage $P^\w$ and $f^\w$ to be able to pass them to $\elim_\IIR$. For the former, we have
\begin{alignat*}{4}
  &\floord{P^\w} : \{t\} \to \IR^\w\,t \to \Set_k\\
  &\floord{P^\w}\,\{t\}\,t^\w := P^\w\,x^\w\,(\elim\,S^*\,P\,f\,t).
\end{alignat*}
For the latter, we first define decoding for induction hypotheses, by induction
on $S^\w$:
\begin{alignat*}{4}
  &\ola{(S^\w)_\IH} &&\,\,:&&(\floord{S^\w}\,p)_\IH\,\floord{P^\w}\,t^\w\\
  & &&\to\,\,&&(S^\w)_{\IH^\w}\,(\snd\,(\snd\,(\ola{(S^\w)_0}\,p\,t^\w)))\,(S_\map\,(\elim\,S^*\,P\,f)\,(\fst\,(\ola{(S^\w)_0}\,p\,t^\w)))
\end{alignat*}
And define
\begin{alignat*}{4}
  &\floord{f^\w} : \{t\}(t^\w : \PSbe_0\,\IR^\w\,\El^\w\,t) \to \PSbe_\IH\,\floord{P^\w}\,t^\w \to \floord{P^\w}\,(\intro\,t^\w)\\
  &\floord{f^\w}\,t^\w\,\ih^\w := f^\w\,(\snd\,(\snd\,(\ola{(S^\w)_0}\,p\,t^\w)))\,(\ola{(S^{*\w})_{\IH}}\,\,\here\,\ih^\w).
\end{alignat*}
Hence, elimination is interpreted as follows:
\[ \elim^\w\,t^\w := \elim_\IIR\,\PSbe\,\floord{P^\w}\,\floord{f^\w}\,t^\w \]

\paragraph{Interpretation of $\elimbeta$.}
The goal is the following:
\[ \elimbeta^\w : \{t\}(t^\w : (\Ssw)_{0^\w}\,t) \to \elim^\w\,(\intro^\w\,t^\w) = f^\w\,t^\w\,((\Ssw)_{\map^\w}\,\elim^\w\,t^\w) \]


        %% elimβᴾ : ∀ {x : F0 S*} (xᴾ : F0ᴾ S*ᴾ x)
        %%          → elimᴾ (wrapᴾ xᴾ) ≡ metᴾ xᴾ (mapIHᴾ S*ᴾ xᴾ elimᴾ)
        %% elimβᴾ {x} xᴾ =


\section{TODO}


\begin{itemize}
  \item Agda sync: rename wrap to intro, unit type in every univ, naming/notation, postulate a Tm universe for IRCanonicity.
  \item metatheory: Loic, Anton say it looks OK
\end{itemize}

\bibliographystyle{ACM-Reference-Format}
\bibliography{references}

\end{document}
\endinput
