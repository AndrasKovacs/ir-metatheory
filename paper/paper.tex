
%% build: latexmk -pdf -pvc paper.tex

\documentclass[acmsmall,screen,review,anonymous]{acmart}
%% \documentclass[nonacm,acmsmall]{acmart}
%% \documentclass[acmsmall]{acmart}
%% \raggedbottom

%% \BibTeX command to typeset BibTeX logo in the docs
\AtBeginDocument{%
  \providecommand\BibTeX{{%
    Bib\TeX}}}

%% Rights management information.  This information is sent to you
%% when you complete the rights form.  These commands have SAMPLE
%% values in them; it is your responsibility as an author to replace
%% the commands and values with those provided to you when you
%% complete the rights form.
\setcopyright{acmlicensed}
\copyrightyear{2018}
\acmYear{2018}
\acmDOI{XXXXXXX.XXXXXXX}

%%
%% These commands are for a JOURNAL article.
\acmJournal{JACM}
\acmVolume{37}
\acmNumber{4}
\acmArticle{111}
\acmMonth{8}

%%
%% The majority of ACM publications use numbered citations and
%% references.  The command \citestyle{authoryear} switches to the
%% "author year" style.
%%
%% If you are preparing content for an event
%% sponsored by ACM SIGGRAPH, you must use the "author year" style of
%% citations and references.
%% Uncommenting
%% the next command will enable that style.
\citestyle{acmauthoryear}

%% --------------------------------------------------------------------------------

\usepackage{xcolor}
\usepackage{mathpartir}
\usepackage{todonotes}
\presetkeys{todonotes}{inline}{}
\usepackage{scalerel}
\usepackage{bm}
\usepackage{mathtools}
\usepackage{stmaryrd}
\usepackage[title]{appendix}

\newcommand{\mit}[1]{{\mathsf{#1}}}
\newcommand{\msf}[1]{{\mathsf{#1}}}
\newcommand{\mbf}[1]{{\mathbf{#1}}}
\newcommand{\mbb}[1]{\mathbb{#1}}
\newcommand{\data}{\mbf{data}}
\newcommand{\U}{\msf{U}}
\newcommand{\Set}{\msf{Set}}
\newcommand{\where}{\mbf{where}}
\newcommand{\Nat}{\msf{Nat}}
\newcommand{\El}{\msf{El}}


%% \newcommand{\rel}{^{\approx}}

%% \newcommand{\Id}{\msf{Id}}
%% \newcommand{\p}{\mathsf{p}}
%% \newcommand{\q}{\mathsf{q}}
%% \newcommand{\code}{\mathsf{code}}
%% \newcommand{\bs}[1]{\boldsymbol{#1}}
%% \newcommand{\wh}[1]{\widehat{#1}}
%% \newcommand{\mdo}{\mbf{do}\,}
%% \newcommand{\ind}{\hspace{1em}}
%% \newcommand{\bif}{\mbf{if}\,}
%% \newcommand{\bthen}{\mbf{then}\,}
%% \newcommand{\belse}{\mbf{else}\,}
%% \newcommand{\return}{\mbf{return}\,}
%% \newcommand{\pure}{\mbf{pure}\,}
%% \newcommand{\lam}{\lambda\,}
%% \newcommand{\data}{\mbf{data}\,}
%% \newcommand{\where}{\mbf{where}}
%% \newcommand{\M}{\msf{M}}
%% \newcommand{\letrec}{\mbf{letrec}\,}

%% \newcommand{\of}{\mbf{of}\,}
%% \newcommand{\go}{\mit{go}}
%% \newcommand{\add}{\mit{add}}
%% \newcommand{\letdef}{\mbf{let\,}}
%% \newcommand{\map}{\mit{map}}
%% \newcommand{\emptycon}{\scaleobj{.75}\bullet}
%% \newcommand{\Tyo}{\msf{Ty}_{\mbbo}}
%% \newcommand{\Tmo}{\msf{Tm}_{\mbbo}}
%% \newcommand{\Cono}{\msf{Con}_{\mbbo}}
%% \newcommand{\Subo}{\msf{Sub}_{\mbbo}}
%% \newcommand{\whset}{\wh{\Set}}
%% \newcommand{\ev}{\mbb{E}}
%% \newcommand{\re}{\mbb{R}}
%% \newcommand{\welim}{\vW{-}\msf{elim}}

%% \newcommand{\mbbc}{\mbb{C}}
%% \newcommand{\mbbo}{\mbb{O}}

%% \newcommand{\vas}{\mathsf{as}}
%% \newcommand{\vbs}{\mathsf{bs}}
%% \newcommand{\vcs}{\mathsf{cs}}
%% \newcommand{\vxs}{\mathsf{xs}}
%% \newcommand{\vys}{\mathsf{ys}}
%% \newcommand{\vsp}{\mathsf{sp}}
%% \newcommand{\vma}{\mathsf{ma}}
%% \newcommand{\vga}{\mathsf{ga}}
%% \newcommand{\vm}{\mathsf{m}}
%% \newcommand{\vn}{\mathsf{n}}
%% \newcommand{\vk}{\mathsf{k}}
%% \newcommand{\vA}{\mathsf{A}}
%% \newcommand{\vB}{\mathsf{B}}
%% \newcommand{\vC}{\mathsf{C}}
%% \newcommand{\vS}{\mathsf{S}}
%% \newcommand{\vF}{\mathsf{F}}
%% \newcommand{\vR}{\mathsf{R}}
%% \newcommand{\vM}{\mathsf{M}}
%% \newcommand{\vmb}{\mathsf{mb}}
%% \newcommand{\mAs}{\mathsf{As}}
%% \newcommand{\va}{\mathsf{a}}
%% \newcommand{\vb}{\mathsf{b}}
%% \newcommand{\vc}{\mathsf{c}}
%% \newcommand{\vd}{\mathsf{d}}
%% \newcommand{\vx}{\mathsf{x}}
%% \newcommand{\vy}{\mathsf{y}}
%% \newcommand{\vz}{\mathsf{z}}
%% \newcommand{\vf}{\mathsf{f}}
%% \newcommand{\vfs}{\mathsf{fs}}
%% \newcommand{\vg}{\mathsf{g}}
%% \newcommand{\vh}{\mathsf{h}}
%% \newcommand{\vt}{\mathsf{t}}
%% \newcommand{\vs}{\mathsf{s}}
%% \newcommand{\vr}{\mathsf{r}}
%% \newcommand{\vu}{\mathsf{u}}
%% \newcommand{\vl}{\mathsf{l}}
%% \newcommand{\vns}{\mathsf{ns}}
%% \newcommand{\vW}{\mathsf{W}}
%% \newcommand{\vsup}{\mathsf{sup}}
%% \newcommand{\vid}{\mathsf{id}}
%% \newcommand{\whW}{\wh{\vW}}


%% \newcommand{\SOP}{\msf{SOP}}
%% \newcommand{\El}{\msf{El}}
%% \newcommand{\USOP}{\msf{U}_{\msf{SOP}}}
%% \newcommand{\Uprod}{\msf{U_P}}
%% \newcommand{\Elprod}{\msf{El_{P}}}
%% \newcommand{\IsSOP}{\msf{IsSOP}}
%% \newcommand{\forEach}{\msf{forEach}}
%% \newcommand{\single}{\msf{single}}
%% \newcommand{\msplit}{\msf{split}}
%% \newcommand{\mapGen}{\msf{mapGen}}
%% \newcommand{\genPull}{\msf{gen_{Pull}}}
%% \newcommand{\Set}{\msf{Set}}
%% \newcommand{\casePull}{\msf{case_{Pull}}}
%% \newcommand{\appull}{\ap_{\Pull}}

%% \newcommand{\Con}{\msf{Con}}
%% \newcommand{\Sub}{\msf{Sub}}
%% \newcommand{\Tm}{\msf{Tm}}

%% \newcommand{\ext}{\triangleright}

%% \newcommand{\Int}{\msf{Int}}
%% \newcommand{\List}{\msf{List}}
%% \newcommand{\Tree}{\msf{Tree}}
%% \newcommand{\Node}{\msf{Node}}
%% \newcommand{\Leaf}{\msf{Leaf}}
%% \newcommand{\Nil}{\msf{Nil}}
%% \newcommand{\Cons}{\msf{Cons}}
%% \newcommand{\Reader}{\msf{Reader}}
%% \newcommand{\ReaderT}{\msf{ReaderT}}
%% \newcommand{\Monad}{\msf{Monad}}
%% \newcommand{\Applicative}{\msf{Applicative}}
%% \newcommand{\class}{\msf{class}}
%% \newcommand{\Functor}{\msf{Functor}}
%% \newcommand{\Bool}{\msf{Bool}}
%% \newcommand{\Statel}{\msf{State}}
%% \newcommand{\fro}{\leftarrow}
%% \newcommand{\case}{\mbf{case\,}}
%% \newcommand{\foldr}{\msf{foldr}}
%% \newcommand{\foldl}{\msf{foldl}}
%% \newcommand{\rep}{\msf{rep}}
%% \newcommand{\concatMap}{\msf{concatMap}}

%% \newcommand{\Lift}{{\Uparrow}}
%% \newcommand{\Up}{{\Uparrow}}
%% \newcommand{\spl}{{\bs{\sim}}}
%% \newcommand{\ql}{{\bs{\langle}}}
%% \newcommand{\qr}{{\bs{\rangle}}}
%% \newcommand{\bind}{\mathbin{>\!\!>\mkern-6.7mu=}}

%% \newcommand{\MTy}{\msf{MetaTy}}
%% \newcommand{\MTm}{\msf{MetaTm}}
%% \newcommand{\VTy}{\msf{ValTy}}
%% \newcommand{\Ty}{\msf{Ty}}
%% \newcommand{\CTy}{\msf{CompTy}}
%% \newcommand{\True}{\msf{True}}
%% \newcommand{\False}{\msf{False}}
%% \newcommand{\fst}{\msf{fst}}
%% \newcommand{\snd}{\msf{snd}}

%% \newcommand{\blank}{{\mathord{\hspace{1pt}\text{--}\hspace{1pt}}}}

%% \newcommand{\Nat}{\msf{Nat}}
%% \newcommand{\Zero}{\msf{Zero}}
%% \newcommand{\Suc}{\msf{Suc}}
%% \newcommand{\Maybe}{\msf{Maybe}}
%% \newcommand{\MaybeT}{\msf{MaybeT}}
%% \newcommand{\Nothing}{\msf{Nothing}}
%% \newcommand{\Just}{\msf{Just}}

%% \theoremstyle{remark}
%% \newtheorem{notation}{Notation}
%% \newtheorem*{axiom}{Axiom}

%% \newcommand{\id}{\mit{id}}
%% \newcommand{\mup}{\mbf{up}}
%% \newcommand{\mdown}{\mbf{down}}
%% \newcommand{\tyclass}{\mbf{class}}
%% \newcommand{\instance}{\mbf{instance}\,}
%% \newcommand{\Improve}{\msf{Improve}}
%% \newcommand{\Gen}{\msf{Gen}}
%% \newcommand{\unGen}{\mit{unGen}}
%% \renewcommand{\Vec}{\msf{Vec}}
%% \newcommand{\gen}{\mit{gen}}
%% \newcommand{\genRec}{\mit{genRec}}
%% \newcommand{\fmap}{<\!\!\$\!\!>}
%% \newcommand{\ap}
%% \newcommand{\runGen}{\mit{runGen}}
%% \newcommand{\qt}[1]{\ql#1\qr}
%% \newcommand{\lift}{\mit{lift}}
%% \newcommand{\liftGen}{\mit{liftGen}}
%% \newcommand{\MonadGen}{\msf{MonadGen}}
%% \newcommand{\MonadState}{\msf{MonadState}}
%% \newcommand{\MonadReader}{\msf{MonadReader}}
%% \newcommand{\RA}{\Rightarrow}
%% \newcommand{\EitherT}{\msf{EitherT}}
%% \newcommand{\Either}{\msf{Either}}
%% \newcommand{\Left}{\msf{Left}}
%% \newcommand{\Right}{\msf{Right}}
%% \newcommand{\StateT}{\msf{StateT}}
%% \newcommand{\Identity}{\msf{Identity}}

%% \newcommand{\Stop}{\msf{Stop}}
%% \newcommand{\Skip}{\msf{Skip}}
%% \newcommand{\Yield}{\msf{Yield}}

%% \newcommand{\runIdentity}{\mit{runIdentity}}
%% \newcommand{\runReaderT}{\mit{runReaderT}}
%% \newcommand{\newtype}{\mbf{newtype}\,}
%% \newcommand{\runMaybeT}{\mit{runMaybeT}}
%% \newcommand{\runStateT}{\mit{runStateT}}
%% \newcommand{\runState}{\mit{runState}}
%% \newcommand{\dlr}{\,\$\,}

%% \newcommand{\ImproveF}{\msf{ImproveF}}
%% \newcommand{\ExceptT}{\msf{ExceptT}}
%% \newcommand{\State}{\msf{State}}
%% \newcommand{\SumVS}{\msf{SumVS}}
%% \newcommand{\ProdCS}{\msf{ProdCS}}
%% \newcommand{\Here}{\msf{Here}}
%% \newcommand{\There}{\msf{There}}
%% \newcommand{\IsSumVS}{\msf{IsSumVS}}
%% \newcommand{\MonadJoin}{\msf{MonadJoin}}
%% \newcommand{\Stream}{\msf{Stream}}
%% \newcommand{\join}{\mit{join}}
%% \newcommand{\modify}{\mit{modify}}
%% \newcommand{\get}{\mit{get}}
%% \newcommand{\mput}{\mit{put}}
%% \newcommand{\Rep}{\mit{Rep}}
%% \newcommand{\encode}{\mit{encode}}
%% \newcommand{\decode}{\mit{decode}}
%% \newcommand{\mindex}{\mit{index}}
%% \newcommand{\mtabulate}{\mit{tabulate}}
%% \newcommand{\States}{\mit{States}}
%% \newcommand{\seed}{\mit{seed}}
%% \newcommand{\step}{\mit{step}}
%% \newcommand{\Step}{\msf{Step}}
%% \newcommand{\Pull}{\msf{Pull}}
%% \newcommand{\MkPull}{\msf{MkPull}}

%% --------------------------------------------------------------------------------

%%
%% end of the preamble, start of the body of the document source.
%% \hypersetup{draft}
\begin{document}


\title{Canonicity for Indexed Inductive-Recursive Types}

\author{András Kovács}
\orcid{0000-0002-6375-9781}
\affiliation{%
  \institution{University of Gothenburg \& Chalmers University of Technology}
  \city{Gothenburg}
  \country{Sweden}
}
\email{andrask@chalmers.se}


\begin{abstract}
We prove canonicity for a Martin-Löf type theory that supports a countable universe hierarchy where
each universe supports indexed inductive-recursive (IIR) types. We proceed in two steps. First, we
construct IIR types from inductive-recursive (IR) types and intensional identity types, in order to
simplify the subsequent canonicity proof. The constructed IIR types support the same definitional
computation rules that are available in Agda's native IIR implementation. Second, we give a
canonicity proof for IR types, building on the well-known method of Artin gluing. The main idea is
to encode the canonicity predicate for each IR type using a metatheoretic IIR type. In short, we reduce IIR types to IR types, then use metatheoretic IIR types to prove canonicity for IR types.
\end{abstract}

%% \begin{CCSXML}
%% \end{CCSXML}
%% \ccsdesc[500]{Theory of computation~Type theory}
%% \ccsdesc[500]{Software and its engineering~Source code generation}
%% \keywords{}

\maketitle

\section{Introduction}\label{sec:introduction}

Induction-recursion (IR) was first used by Martin-Löf in an informal way \cite{TODO}, then made
formal by Dybjer and Setzer \cite{TODO}, who also developed set-theoretic and categorical semantics
\cite{TODO}. A common application of IR is to define custom universe hierarchies inside a type
theory. In the proof assistant Agda, we can use IR to define a universe that is closed under our
choice of type formers:
\begin{alignat*}{3}
  & \mbf{mutual} \\
  & \quad\data\,\U : \Set\,\where \\
  & \quad\quad \Nat' : \U\\
  & \quad\quad \Pi' \hspace{0.8em} : (A : \U) \to (\El\,A \to \U) \to \U
  & \\
  & \\
  & \quad\El : \U \to \Set \\
  & \quad\El\,\Nat'\hspace{1.5em}  = \Nat \\
  & \quad\El\,(\Pi'\,A\,B) = (a : \El\,A) \to \El\,(B\,a)
\end{alignat*}
This $\U$, unlike the ambient $\Set$ universe, supports an induction principle and can be used to
define type-generic functions. \emph{Indexed induction-recursion} (IIR) additionally allows indexing
$\U$ over some type, which lets us define inductive-recursive predicates \cite{TODO}.

One application of IR has been to develop semantics for object theories that support universe
hierarchies. IR has been used in normalization proofs \cite{TODO} and in modeling first-class
universe levels \cite{TODO} and proving canonicity for them \cite{TODO}. Another application is to
do generic programming over universes of type descriptions \cite{TODO} or data layout descriptions
\cite{TODO}.

IIR has been supported in Agda 2 since the early days of the system \cite{TODO}, and it is also
available in Idris 1 and Idris 2 \cite{TODO}. In these systems, IR has been implemented in the
``obvious'' way, supporting closed program execution in compiler backends and normalization during
type checking, but without any formal justification.

Our \textbf{main contribution} is to \textbf{show canonicity} for a Martin-Löf type theory that
supports a countable universe hierarchy, where each universe supports indexed inductive-recursive
types. Canonicity means that every closed term is definitionally equal to a canonical
term. Canonical terms are built only from constructors; for instance, a canonical natural number
term is a numeral. Hence, canonicity justifies evaluation for closed terms. The outline of our
development is as follows.

%% \begin{itemize}
%% \item For running closed programs, \emph{canonicity} is the most relevant property to show. In
%%   means that every closed term is definitionally equal to a canonical term. Canonical terms are
%%   built only from constructors; for instance, a canonical natural number term is a numeral.
%% \item \emph{Normalization} means that every open term is definitionally equal to a normal form. The
%%   characterizing property of normal forms is that definitional equality coincides with simple
%%   structural equality. Hence, normalization is practically important for deciding definitional
%%   equality, which is in turn required in type checkers for dependently typed languages.
%% \end{itemize}
%% Our \textbf{main contribution} is to show canonicity for a Martin-Löf type theory that supports
%% countable universes with indexed inductive-recursive types in each universe.


\begin{enumerate}
\item In Section \ref{TODO} we specify what it means to support IR and IIR, using Dybjer and
  Setzer's rules with minor modifications \cite{TODO}. We use \emph{first-class signatures},
  meaning that descriptions of (I)IR types are given as ordinary inductive types internally.
\item In Section \ref{TODO} we construct IIR types from IR types and other basic type formers. This
  allows us to only consider IR types in the subsequent canonicity proof, which is a significant
  simplification. In the construction of IIR types, we lose some definitional equalities when IIR
  signatures are neutral, but we still get strict computation for canonical signatures. This matches
  the computational behavior of Agda and Idris, where IIR signatures are second-class and
  necessarily canonical.
\item In Section \ref{TODO}, we give a proof-relevant logical predicate interpretation of the type
  theory, from which canonicity follows. We follow the well-known method of Artin gluing
  \cite{TODO}. The main challenge here is to give a logical predicate interpretation of IR types. We
  do this by using IIR in the metatheory: from each object-theoretic signature we compute a
  metatheoretic IIR signature which encodes the canonicity predicate for the corresponding IR type.
  We formalize the predicate interpretation of IR types in Agda, using a shallow embedding of the
  syntax of the object theory. Hence, there is a gap between the Agda version and the fully formal
  construction, but we argue that it is a fairly modest gap.

\end{enumerate}

\section{Specification for (I)IR types}\label{sec:specification}

















%% IIR has been used in normalization proofs for

%% For example, IR
%% has been used to prove normalization \cite{TODO} and to build semantics for first-class universe
%% levels \cite{TODO} and to prove canonicity for them \cite{TODO}. Another application is in generic
%% programming, where IR is used to define custom universes of datatype descriptions \cite{TODO}.



\bibliographystyle{ACM-Reference-Format}
\bibliography{references}

\end{document}
\endinput
